%!TEX encoding = UTF-8 Unicode
%!TEX root = ../thesis_main.tex


%%%%%%%%%%%%%%%%%%%%%%%%%%%%%%%%%%%%%%%%%%%%%%%%%%%%%%%%%%%%%%%%%%%%
%%                                                                %%
%%                        AUTHOR AND TITLE                        %%
%%                                                                %%
%%%%%%%%%%%%%%%%%%%%%%%%%%%%%%%%%%%%%%%%%%%%%%%%%%%%%%%%%%%%%%%%%%%%

\newcommand{\mytitle}{Robust optimisation of the pathway\texorpdfstring{\\ towards a sustainable whole-energy system}{}}
\newcommand{\mysubtitle}{A hierarchical multi-objective\texorpdfstring{\\ reinforcement-learning based approach}{}}
\newcommand{\myauthor}{Xavier Rixhon}

\def\eg{e.g.\ }
\def\ie{i.e.\ }
\def\og{``}
\def\fg{''}


\usepackage[record,style=alttreegroup,nomain,symbols]{glossaries-extra}
\glsdisablehyper
%!TEX encoding = UTF-8 Unicode
%!TEX root = ../thesis_main.tex

%%%%%%%%%%%%%%%%%%%%%%%%%%%%%%
%	MATH SYMBOLS							%
%%%%%%%%%%%%%%%%%%%%%%%%%%%%%%

\glsxtrsetgrouptitle{Abreviation}{Abreviation}
\glsxtrsetgrouptitle{Notation}{Notation}

% \newcommand{\sdot}{{\bullet}}
\newcommand{\sdot}{(\cdot)}

% ---------------------------------------------------------------------------- %
% ACRONYMS
%% A

%% P
\newabbreviation[group={Acronyms}]{PCE}{PCE}{Polynomial Chaos Expansion}

%% S
\newabbreviation[group={Acronyms}]{SDGs}{SDGs}{Sustainable Development Goals}

%% U
\newabbreviation[group={Acronyms}]{UQ}{UQ}{Uncertainty Quantification}

% ---------------------------------------------------------------------------- %
% ROMAN
%% A
%\glsxtrnewsymbol[group={rs},description={ith ambient particle}]{ai}{\ensuremath{A_i}}


% ---------------------------------------------------------------------------- %
% GREEK
%\glsxtrnewsymbol[group={gs},description={Yaw angle [$\deg$]}]{beta}{\ensuremath{\beta}}


% ---------------------------------------------------------------------------- %
% Sub- Superscripts
%% A
%\glsxtrnewsymbol[group={ss},description={Relative to the ambient}]{uA}{\ensuremath{\sdot_{A}}}



\renewcommand\glstreegroupheaderfmt[1]{\begingroup\textbf{#1}\vspace{-.2cm}\par\endgroup}
\glsfindwidesttoplevelname
\glsxtrsetgrouptitle{Acronyms}{Acronyms}
\glsxtrsetgrouptitle{rs}{Roman Symbols}
\glsxtrsetgrouptitle{gs}{Greek Symbols}
\glsxtrsetgrouptitle{ss}{Sub- Superscripts}
\glsxtrsetgrouptitle{o}{Operators}

%------------------------
%% A
%\newcommand{\ABL}[2]{$U_\mABL=\SIms{#1}$ - $TI_\mABL={#2}\%$}
%------------------------
%% B

%------------------------
%% C

%------------------------
%% D
\newcommand{\degree}{^\circ}
\newcommand{\dd}{\mathrm{d}}

%------------------------
%% E
\newcommand{\etal}{\textit{et~al.}}
\newcommand{\eg}{e.g.\ }

%------------------------
%% F

%------------------------
%% G

%------------------------
%% H

%------------------------
%% I
\newcommand{\ie}{i.e.\ }

%------------------------
%% K

%------------------------
%% L

%------------------------
%% M

%------------------------
%% N

%------------------------
%% O

%------------------------
%% P

%------------------------
%% Q

%------------------------
%% R

%------------------------
%% S

%------------------------
%% T


%------------------------
%% U
\newcommand{\uh}{\hat{u}}
\newcommand{\uv}{\mathbf{u}} 

\newcommand{\ub}{\boldsymbol{u}} 

\newcommand{\uconvw}{\tilde{U}_W}

%------------------------
%% V
\newcommand{\vv}{\mathbf{v}}
\newcommand{\vb}{\boldsymbol{v}}

%------------------------
%% W
\newcommand{\wv}{\mathbf{w}}
\newcommand{\wb}{\boldsymbol{w}}
\newcommand{\WT}{{\scriptscriptstyle \mathrm{WT}}}

%------------------------
%% X
\newcommand{\xv}{\mathbf{x}}
\newcommand{\xb}{\boldsymbol{x}}

%------------------------
%% Y
\newcommand{\yv}{\mathbf{y}}
\newcommand{\yb}{\boldsymbol{y}}

%------------------------
%% Z
\newcommand{\zv}{\mathbf{z}}
\newcommand{\zb}{\boldsymbol{z}}

%%%%%%%%%%%%%%%%%%%%%%%%%%%%%%
%	OPERATORS 								%
%%%%%%%%%%%%%%%%%%%%%%%%%%%%%%
\newcommand{\der}[2]{\frac{d #1}{d #2}}
\newcommand{\pder}[2]{\frac{\partial #1}{\partial #2}}
\newcommand{\pdern}[3]{\frac{\partial^#3 #1}{\partial #2^#3}}

\newcommand{\avgt}[1]{\left\langle {#1}\right\rangle}
\newcommand{\avgs}[1]{\overline{#1}}

%%%%%%%%%%%%%%%%%%%%%%%%%%%%%%
%	REFERENCES 							%
%%%%%%%%%%%%%%%%%%%%%%%%%%%%%%

\newcommand{\eqqref}[1]{Eq.~\eqref{#1}}
\newcommand{\seqqref}[1]{Eqs.~\eqref{#1}}
\newcommand{\eqqrefs}[2]{Eqs.~\eqref{#1} and \eqref{#2}}
\newcommand{\eqqrefc}[2]{Eqs.~\eqref{#1}, \eqref{#2}}
\newcommand{\eq}[1]{Eq.~\eqref{#1}}
% \newcommand{\eqs}[1]{Eqs.~(\ref{#1})}

\newcommand{\fig}{Fig.~\ref}
\newcommand{\figs}{Figs.~\ref}
\newcommand{\tab}{Table~\ref}
\newcommand{\sect}{Section~\ref}
\newcommand{\chap}{Chapter~\ref}
\newcommand{\chaps}{Chapters~\ref}
\newcommand{\app}{Appendix~\ref}
\newcommand{\apps}{Appendices~\ref}

%%%%%%%%%%%%%%%%%%%%%%%%%%%%%%
%	CAPTIONS							%
%%%%%%%%%%%%%%%%%%%%%%%%%%%%%%
% Color palette
  
\usepackage{xcolor}
\definecolor{myBlue}{rgb}{0.25, 0.25, 0.75}
\definecolor{myRed}{rgb}{0.8, 0.1, 0.3}
\definecolor{myGrey0}{rgb}{0.25, 0.25, 0.25}
\definecolor{myGrey1}{rgb}{0.50, 0.50, 0.50}
\definecolor{myGrey2}{rgb}{0.75, 0.75, 0.75}


\definecolor{Blue0}{rgb}{0.7752402921953095, 0.8583006535947711, 0.9368242983467897}
\definecolor{Blue1}{rgb}{0.41708573625528644, 0.6806305267204922, 0.8382314494425221}
\definecolor{Blue2}{rgb}{0.1271049596309112, 0.4401845444059977, 0.7074971164936563}
\definecolor{Blue3}{rgb}{0.03137254901960784, 0.18823529411764706, 0.4196078431372549}

\definecolor{BluePy}{rgb}{0.69140625, 0.8164, 0.8868}
\definecolor{RedC}{rgb}{0.9453, 0.7266, 0.6602}

\definecolor{gw}{rgb}{1, 0, 0}
\definecolor{jb}{rgb}{0, 1, 0}
\definecolor{tg}{rgb}{0, 0, 1}
\definecolor{jh}{rgb}{1, 1, 0}

\newcommand{\gw}[1]{{#1}}
% \newcoxmmand{\gw}[1]{\textcolor{gw}{#1}}
\newcommand{\jb}[1]{\textcolor{jb}{#1}}
\newcommand{\tg}[1]{\textcolor{tg}{#1}}
\newcommand{\jh}[1]{\textcolor{jh}{#1}}


%%%%%%%%%%%%%%%%%%%%%%%%%%%%%%%%%%%%%%%%%%%%%%%%%%%%%%%%%%%%%%%%%%%%
%%                                                                %%
%%                       LANGUAGE AND FONTS                       %%
%%                                                                %%
%%%%%%%%%%%%%%%%%%%%%%%%%%%%%%%%%%%%%%%%%%%%%%%%%%%%%%%%%%%%%%%%%%%%

% -- Language ------------------------------------------------------
\usepackage[utf8]{inputenc}
\usepackage[USenglish]{babel}
\usepackage[T1]{fontenc} % Accents
\usepackage{scrextend} % to use \footref: multiple reference to the same table footnote
\usepackage{pbox} % to have new line inside table cells

% -- Fonts ---------------------------------------------------------
%\usepackage{lmodern}
%% utopia
%\usepackage{fourier}
%% palatino
%\usepackage{palatino}
%% pifont
%\usepackage{pifont}
%% charter
%\usepackage{charter}

% stix
\usepackage[lcgreekalpha]{stix}
\usepackage{textcomp}

%% better and newer palatino font
%\usepackage{newpxtext,newpxmath}

%% special font, no ams package included...
%\usepackage[full]{textcomp}
%\usepackage[osf]{newtxtext} % osf for text, not math
%\usepackage{cabin} % sans serif
%\usepackage[varqu,varl]{inconsolata} % sans serif typewriter
%\usepackage[bigdelims,vvarbb]{newtxmath} % bb from STIX
%\usepackage[cal=boondoxo]{mathalfa} % mathcal


%%%%%%%%%%%%%%%%%%%%%%%%%%%%%%%%%%%%%%%%%%%%%%%%%%%%%%%%%%%%%%%%%%%%
%%                                                                %%
%%                    TEXT, LAYOUT AND STRUCTURE                  %%
%%                                                                %%
%%%%%%%%%%%%%%%%%%%%%%%%%%%%%%%%%%%%%%%%%%%%%%%%%%%%%%%%%%%%%%%%%%%%

% -- Paragraphs---- ------------------------------------------------
\linespread{1.15}
%\usepackage[parfill]{parskip} %% NOT working with memoir class... see https://tex.stackexchange.com/questions/193485/how-to-use-usepackageparfillparskip-in-memoir-class for discussion

% -- Margins -------------------------------------------------------
\usepackage[includehead,includefoot,centering,height=20cm,width=12cm,showframe=true]{geometry}
%\usepackage[includehead,includefoot,centering,height=20cm,width=12cm,showframe=false]{geometry}

%\usepackage[includehead,includefoot,centering,height=20cm,width=12cm,showframe=true]{geometry} %taille des marges
%\usepackage[includehead, includefoot, top=4.2cm, bottom=5.4cm, left=4.5cm, right=4.5cm,showframe=true]{geometry}%…PhP
%\usepackage[includehead, includefoot, top=4.8cm, bottom=4.8cm, left=4.5cm, right=4.5cm,showframe=true]{geometry}%4.8 4.8

% -- Layout --------------------------------------------------------
%\usepackage{layout} % pour afficher le layout des pages

%\usepackage{color}
%\usepackage[dvipsnames]{xcolor}s
\usepackage{lineno}
%\usepackage{ulem} %underline for emphasis
\usepackage{fancyhdr}% pour en-tetes personnalises
\usepackage{appendix} % gestion appendix
\usepackage{tikz}% pour diagrammes
\usetikzlibrary{shapes.misc}
\usepackage{setspace}

%\usepackage{titlepages} % for the titlepage examples

%\usepackage{charter}
%\usepackage[raggedright]{titlesec} % evite les coupures de mot dans les titres

\usepackage{afterpage} % Force content to be on the next page

%% The lineno packages adds line numbers. Start line numbering with
%% \begin{linenumbers}, end it with \end{linenumbers}. Or switch it on
%% for the whole article with \linenumbers.
\usepackage{lineno} % Line numbers

\sloppy % evite les debordements de texte dans la marge
\raggedbottom %regroupe les espaces verticaux automatiques au bas des pages

% -- Structure et Paragraphes --------------------------------------
%evite les lignes veuves et orphelines
\clubpenalty=10000
\widowpenalty=10000
\interfootnotelinepenalty=10000

%\setcounter{secnumdepth}{3} % numerote les subsubsections
\setsecnumdepth{subsection}
\maxtocdepth{subsection}

% -- Style ---------------------------------------------------------
\normalfont
% chapter style
%\chapterstyle{ell}
%\chapterstyle{madsen}
%\chapterstyle{bianchi}
%\chapterstyle{dash}
%\chapterstyle{thatcher}
\newcommand{\clearemptydoublepage}{\newpage{\pagestyle{empty}\cleardoublepage}} %pour effacer les en-tetes sur la page vierge avant chaque chapitre

%\pagestyle{ruled}
\nouppercaseheads
\pagestyle{headings}
\makeheadrule{headings}{\textwidth}{0.3pt}


%%%%%%%%%%%%%%%%%%%%%%%%%%%%%%%%%%%%%%%%%%%%%%%%%%%%%%%%%%%%%%%%%%%%
%%                                                                %%
%%                     TABLES AND FIGURES                         %%
%%                                                                %%
%%%%%%%%%%%%%%%%%%%%%%%%%%%%%%%%%%%%%%%%%%%%%%%%%%%%%%%%%%%%%%%%%%%%

% -- Figures -------------------------------------------------------
\usepackage{graphicx}
\graphicspath{{frontend/img/}{chap_intro_ccl/img/}{chap_case_study/img/}{chap_ES_PCE/img/}{chap_electro_uq/img/}{chap_methodo/img/}{chap_myopic/img/}{chap_RL/img/}{chap_atom_mol/img/}{chap_RobPol/img/}{appendices/img/}} % Figures folder different for each input
%\graphicspath{{img/}} %
\DeclareGraphicsExtensions{.eps,.pdf,.png,.jpg}

%\usepackage[cleanup,process=auto]{pstool}
%\usepackage[cleanup,process=all]{pstool} % force process every figure
%\usepackage{pst-pdf}
%\usepackage{epstopdf}
%\usepackage{auto-pst-pdf} 
%\usepackage{psfrag}
%\PassOptionsToPackage{psfrag}{process=none}

\usepackage[figuresright]{rotating} %https://tex.stackexchange.com/questions/337/how-to-change-certain-pages-into-landscape-portrait-mode

\def\figwidth{0.9\textwidth}
\newcommand{\singlefigwidth}{.75\textwidth}

\setfloatadjustment{figure}{\centering}
\setfloatlocations{figure}{thb}
\newsubfloat{figure}

\newcommand{\illusname}{Illustration}
\newcommand{\listillusname}{List of Illustrations}
\newlistof{listofillus}{loillu}{\listdiagramname}
\newfloat[chapter]{illustration}{loillu}{\illusname}
\newlistentry{illustration}{loillu}{0}
\renewcommand\theillustration{\arabic{illustration}}    

\usetikzlibrary{shapes}

% -- Tables --------------------------------------------------------
\usepackage{booktabs} % Table customization: \specialrule

\usepackage{longtable} % Long tables on multiple pages
\usepackage{tabu}
\usepackage{ltablex} % Use ltablex instead of tabularx
\usepackage{multirow} % Cellule de tableau sur plusieurs lignes
\usepackage{multicol} % Cellule de tableau sur plusieurs colonnes
\usepackage{dcolumn} % Columns aligned on decimal point
\newcolumntype{d}[1]{D{.}{.}{#1}}
\usepackage{array} % ??

%\usepackage{cellspace} % Espace entre le texte et les bord des cellules
%\usepackage{tabularx} % Tableau avec largeur fixée

\usepackage{makecell} % Customize table headers with \thead
\renewcommand\theadfont{\bfseries} % Customize table column headers with \thead

% -- Sub-figures and Captions --------------------------------------
\usepackage{subfig}
\usepackage{wrapfig} % Images dans le texte
%\usepackage{floatrow} % Tableau et figure côte à côte
%\usepackage{subcaption} % sub caption

\usepackage[labelfont=bf, labelsep=period, justification=justified,singlelinecheck=false]{caption} % style et taille de police des captions
%\captionsetup[table]{position=below}

%\captionnamefont{\sffamily} % font for the "Figure xx" part
%\captionnamefont{\footnotesize\sffamily} % font for the "Figure xx" part
%\captionnamefont{\footnotesize} % font for the "Figure xx" part
%\captiontitlefont{\footnotesize} % font for the caption part
%\captiontitlefont{\small}
% change caption witdth
%\changecaptionwidth
%\captionwidth{0.9\textwidth}
%\captionstyle[\centering]{}


%%%%%%%%%%%%%%%%%%%%%%%%%%%%%%%%%%%%%%%%%%%%%%%%%%%%%%%%%%%%%%%%%%%%
%%                                                                %%
%%                      SCIENTIFIC PACKAGES                       %%
%%                                                                %%
%%%%%%%%%%%%%%%%%%%%%%%%%%%%%%%%%%%%%%%%%%%%%%%%%%%%%%%%%%%%%%%%%%%%

% -- Math ----------------------------------------------------------
\usepackage{amsmath}
%\usepackage[intlimits]{amsmath}
\usepackage{amsfonts}
\usepackage{amsthm} % Extended theorem environments
%\usepackage{amssymb} % Math symbols %%redundant with stix package
%\usepackage{esint} % Intégrales multiples
%\usepackage{esvect} % Vecteurs
\usepackage{mathtools}
\usepackage{pifont}

% Put the equation label in the document, to help writing
\usepackage{showlabels}
%\usepackage[final]{showlabels} %disable

\usepackage[boxed]{algorithm2e}

% -- Physics and Chemistry -----------------------------------------
\usepackage{siunitx} % SI units
\usepackage[version=4]{mhchem} % Chemical equations

% -- Symbols -------------------------------------------------------
\usepackage[super]{nth} %text superscript \nth{i}
\usepackage{textcomp} % Degree symbol °, and certainly others
%\newcommand{\cmark}{\ding{51}}
%\newcommand{\xmark}{\ding{55}}


%%%%%%%%%%%%%%%%%%%%%%%%%%%%%%%%%%%%%%%%%%%%%%%%%%%%%%%%%%%%%%%%%%%%
%%                                                                %%
%%                  BIBLIOGRAPHY AND HYPERREF                     %%
%%                                                                %%
%%%%%%%%%%%%%%%%%%%%%%%%%%%%%%%%%%%%%%%%%%%%%%%%%%%%%%%%%%%%%%%%%%%%

% -- Bibliography --------------------------------------------------
% For natbib help, see http://merkel.texture.rocks/Latex/natbib.php?lang=fr
%\usepackage{biblatex}
\usepackage[square,numbers,sort&compress]{natbib}
%\usepackage{pdfcomment}
%\usepackage[square,numbers,compress]{natbibtooltip}
\usepackage{textcomp} % recognize the textquotesingle from bib

\bibliographystyle{options/elsarticle-num-names}  %Ordered by appearance in the text, with DOI and URL

%\bibliographystyle{options/elsarticle-num}  %Same but \citet command does not work
%\bibliographystyle{options/elsarticle-harv}  %Harvard style (Author, year, no number)
%\bibliographystyle{options/elsarticle-num-names-nourldoi}  %Ordered by appearance in the text, with DOI
%\bibliographystyle{options/elsarticle-names-nourldoi}  %Ordered by alphabet, with DOI
%\bibliographystyle{options/elsarticle-names-nourl} %Ordered by alphabet, without DOI
%\bibliographystyle{options/model4-names}

% -- Hyperref ------------------------------------------------------
\PassOptionsToPackage{hyphens}{url} %this helps breaking URL when too long
\usepackage[bookmarks,hidelinks]{hyperref}
\usepackage{cleveref}
\usepackage{bm} % To use \bm in order to get bold math symbols

% PDF metadata
\hypersetup{pdftitle={\mytitle}}
\hypersetup{pdfauthor={\myauthor}}
\hypersetup{pdfsubject={PhD thesis}}
\hypersetup{pdfkeywords={Energy System Modelling,EnergyScope,Reinforcement Learning,Policy Optimisation, Electro-fuels}}

% Links color
\newcommand\myshade{85}
% from http://latexcolor.com
\definecolor{airforceblue}{rgb}{0.36, 0.54, 0.66}
\definecolor{battleshipgrey}{rgb}{0.52, 0.52, 0.51}
\definecolor{burntumber}{rgb}{0.54, 0.2, 0.14}
\definecolor{sangria}{rgb}{0.57, 0.0, 0.04}
\colorlet{mylinkcolor}{sangria}
\colorlet{mycitecolor}{battleshipgrey}

%\hypersetup{
%%linkcolor  = mylinkcolor!\myshade!black,
%%citecolor  = mycitecolor!\myshade!black,
%%urlcolor   = myurlcolor!\myshade!black,
%colorlinks = true,
%}

%\hypersetup{
%    colorlinks=true,
%    citecolor = blue,
%    linkcolor=blue,
%    urlcolor = blue
% }

%% Links with color frame
\hypersetup{
colorlinks = false,
%citebordercolor ={red},
}

%% No color on links (for printing)
%\hypersetup{hidelinks}

% Adding bookmarks to pdf (works after clearing aux and then 2 clean compilations)
%\hypersetup{bookmarks={true}} % --> does it by default

%% Reference with Fancy Tooltips (only works with the dedicated compilation script)
%from: https://tex.stackexchange.com/questions/84681/interactive-pdf-latex-and-article-of-the-future/84700#84700
%\usepackage[inactive,blur=0.6, fixcolor]{fancytooltips}
%\usepackage[inactive]{fancytooltips}

% -- Footnotes ------------------------------------------------------
\newcommand\blfootnote[1]{%
  \begingroup
  \renewcommand\thefootnote{}\footnote{#1}%
  \addtocounter{footnote}{-1}%
  \endgroup} %%Footnotes sans numéro