\documentclass[12pt,a4paper]{article}
\usepackage{longtable}
\usepackage[LGR,T1]{fontenc}
\usepackage{lscape}
%\usepackage{pdfsync}
\usepackage{multirow}
\usepackage{amsmath,bm} 
\usepackage{amsfonts}
\usepackage{amsthm} % Extended theorem environments
%\usepackage{amssymb} % Math symbols %%redundant with stix package
%\usepackage{esint} % Intégrales multiples
%\usepackage{esvect} % Vecteurs
\usepackage{mathtools}
\usepackage{pifont} 
\usepackage[version=4]{mhchem} % Chemical equations
 
\usepackage{fancyhdr}
\usepackage{graphicx}

\graphicspath{{../frontend/img/}{../chap_intro_ccl/img/}{../chap_case_study/img/}{../chap_ES_PCE/img/}{../chap_electro_uq/img/}{../chap_methodo/img/}{../chap_myopic/img/}{../chap_RL/img/}{../chap_atom_mol/img/}{../chap_RobPol/img/}{../appendices/img/}{../rebuttals/img/}} % Figures folder different for each input
%\graphicspath{{img/}} %
\DeclareGraphicsExtensions{.eps,.pdf,.png,.jpg}




\usepackage{lastpage}
\usepackage{afterpage}
\usepackage{lettrine}
\usepackage{color,soul}
\usepackage[dvipsnames]{xcolor}
\usepackage{colortbl}
\usepackage{enumitem}
\usepackage{tikz}
\usepackage{titlesec}
\def\eg{e.g.,\ }
\def\ie{i.e.,\ }

\newcommand{\clearemptydoublepage}{\newpage{\pagestyle{empty}\cleardoublepage}} %pour effacer les en-tetes sur la page vierge avant chaque chapitre


%Palatino font
%\usepackage{pxfonts}
%\usepackage{libertine}
\usepackage[scaled=0.88]{beraserif}
\usepackage[scaled=0.85]{berasans}
\usepackage[scaled=0.84]{beramono}
\usepackage{mathpazo}
%\linespread{1.05}
\usepackage[T1,small,euler-digits]{eulervm}

\usepackage[nomessages]{fp}


\definecolor{bleuUCLclair}{rgb}{.09, 0.569, 1}
\definecolor{bleuUCLfonce}{rgb}{ .13, .52, .86}
\definecolor{redBurn}{rgb}{.91, 0.29, 0.08}

\usepackage{tocloft}

%\renewcommand{\cftsecpresnum}{Reviewer \#}
\renewcommand{\cftsecpresnum}{}
\renewcommand{\cftsecnumwidth}{6em}
\renewcommand{\cftsubsecpresnum}{}
\renewcommand{\cftsubsecnumwidth}{6em}

\setcounter{tocdepth}{3}
\setcounter{secnumdepth}{3}

\usepackage[nottoc]{tocbibind}
\usepackage[colorlinks=true,urlcolor=black,linkcolor=black,citecolor=black]{hyperref}
\usepackage[square,numbers,sort&compress]{natbib}

\bibliographystyle{biblio/elsarticle-num-names}  %Ordered by appearance in the text, with DOI and URL
%\usepackage[backend=biber, natbib=true, style=numeric-comp, citestyle=numeric-comp, sorting=none, giveninits=true, maxcitenames=1]{biblatex}

\usepackage[]{tocbibind}
\usepackage{hyperref}
\hypersetup{
    colorlinks=true,
%    linkcolor=black,
    bookmarks=true,
    pdfpagemode=FullScreen,
}
\setcounter{tocdepth}{2}

\addtolength{\topmargin}{-1.5cm}
\addtolength{\textheight}{1.5cm}
\addtolength{\textwidth}{2cm}
\addtolength{\footskip}{2cm}
\setlength{\evensidemargin}{-0.5cm}
\setlength{\oddsidemargin}{-0.5cm}
\setlength{\arrayrulewidth}{0.25pt}

\renewcommand{\baselinestretch}{1.1} % Interligne


\newenvironment{maliste}%
{ \begin{list}%
	{\textcolor{bleuUCLfonce}{$\bullet$}\hspace{0.5cm}}%
	{\setlength{\labelwidth}{50pt}%
	 \setlength{\leftmargin}{25pt}%
	 \setlength{\itemsep}{30pt}}}%
{ \end{list} }

%\renewcommand{\headrulewidth}{0.0pt}
%\newcommand{\clearemptydoublepage}{%
%	\newpage{\pagestyle{empty}\cleardoublepage}}




%section like title in longtable
\newcommand{\seclong}[1]{\multicolumn{2}{@{}l}{{\Large\sffamily #1}} 
\vspace{0.5cm}
\\}

%enumerate on two columns
\newcounter{listlong}
\newcommand{\newlistlong}{\setcounter{listlong}{1}}
\newcommand{\iteml}[1]{%
\hspace{4.5cm}\textcolor{redBurn}{\arabic{listlong}}\stepcounter{listlong}%
&%
#1%
 
\\%
}

%left in column
\newcommand{\lcol}[1]{%
\begin{minipage}[t]{.35\textwidth}%

#1%

\end{minipage}%
}

\title{\vspace{-1cm}
\begin{flushleft} {\sffamily Xavier Rixhon's PhD thesis - Answers to jury members - SHORT}\end{flushleft}}
\date{\vspace{-1.7cm}\begin{flushleft}\sffamily Exploration of uncertainty-aware energy transition pathways - Reinforcement learning and principal component analysis-based methods\end{flushleft}}
%
%Xavier Rixhon, Gauthier Limpens, Diederik Coppitters, Hervé Jeanmart and Francesco Contino\end{flushleft}}


\pagestyle{fancy} 
\fancyhf{}
\fancyfoot[R]{\sffamily\thepage\ / \pageref{LastPage}}
  \fancyfoot[L]{ }

\fancypagestyle{plain}{%
  \fancyhf{}%
  \fancyfoot[R]{\sffamily\thepage\ / \pageref{LastPage}}
  \fancyfoot[L]{ }
}

\renewcommand{\headrulewidth}{0.0pt}


\newcommand{\hlc}[2][yellow]{ {\sethlcolor{#1} \hl{#2}} }

\titleformat{\section}
  {\bfseries\scshape}{}{1em}{}

\titleformat{\subsection}
  {\normalfont\scshape}{}{1em}{}

\usepackage[framemethod=default]{mdframed}
%\mdfsetup{skipabove=\topskip,skipbelow=\topskip}

\global\mdfdefinestyle{comment}{%
     linecolor=red,linewidth=0.1cm,%
     leftmargin=-0.5cm,rightmargin=-0.5cm, innerleftmargin=0.4cm,innerrightmargin=0.4cm,
     topline=false,bottomline=false
}

\global\mdfdefinestyle{manuscript}{%
     linecolor=gray!20,linewidth=0.05cm,backgroundcolor=gray!20,%
     leftmargin=-0.5cm,rightmargin=-0.5cm, innerleftmargin=0.4cm,innerrightmargin=0.4cm
}
  
  \renewcommand{\subsectionautorefname}{Comment}
\begin{document}
\maketitle

I would like to thank the jury members for their comments, which significantly helped improving the manuscript and substantiating the novelty of my work. Based on the notes taken during the private defense, I hereby answered, as accurately as possible, the jury's comments. 

I believe I have addressed all the issues raised in the following answers. Some of them required adaptations either to the text of the PhD thesis or to the subsequent papers. These adaptations are either directly in the thesis manuscript or left for further developments in subsequent papers. For each comment, I have first highlighted the issue, then provided an answer, and finally described how the manuscript was adjusted, if needed.\\

When a comment explicitly comes from specific members of the jury, they are listed at the beginning of the comment with the following color code: {\color{orange} \textbf{Stefano} Moret}, {\color{teal} \textbf{Stefan} Pfenninger}, {\color{purple} \textbf{Sylvain} Quoilin} and {\color{violet} \textbf{Christophe} De Vleeschouwer}. Here is how an answer to a comment is structured:

\begin{mdframed}[style=comment] % Comment from the reviewer
Text of the comment. When a page number is mentioned at the beginning of the comment, it refers to the page number of the manuscript submitted before the private defense, which might differ from the one in its final version.
\end{mdframed}

\noindent The answer provided to the comment. Based on this, if a modification was brought to the manuscript, it is located in the thesis manuscript using {\color{blue} blue font}.

\begin{mdframed}[style=manuscript] % Modification brought to the manuscript
New version of the text in the manuscript.
\end{mdframed}

% {\color{orange} \textbf{Stefano}}
% {\color{purple} \textbf{Sylvain}}
% {\color{teal} \textbf{Stefan}}
% {\color{violet} \textbf{Christophe}}

\clearpage
\tableofcontents

\clearemptydoublepage

\section{General comments/questions}
\label{General}

\begin{mdframed}[style=comment] % Comment from the reviewer
{\color{orange} \textbf{Stefano}} - Overall, many methods: PCE + RL + NN + PCA. At each stage, some assumption is made. Why not a simpler approach or a simpler model? Do you need all these steps? Explain the role of the different methods and why PCE instead of monthly version for uncertainty? 
\end{mdframed}

\noindent The purpose of each method and the assumptions are discussed in the ``Assumptions - Fine tuning'' section of the present document. Regarding the choice of PCE instead of monthly version for uncertainty, as discussed during the private defense, monthly version of the model does not provide enough detail to build sufficient confidence in the results given the need to massively integrate VRES for Belgium to succeed its transition. Given the hourly/daily intermittency, properly assessing the integration of wind and solar requires a fine time resolution that is not provided by the monthly model. 

Given what is detailed further in this document and what has been discussed during the private defense, I have not brought modification to the manuscript in this regard.

\begin{mdframed}[style=comment] % Comment from the reviewer
{\color{violet} \textbf{Christophe}} - The variables and the uncertainties should be presented in a compact and mathematical way.
\end{mdframed}

\noindent In my opinion, the main variables of EnergyScope Pathway are presented {\color{blue} in Section 1.1 of the core of the manuscript}. Further details are also given {\color{blue}in Appendix B.2}. Finally, these variables are exhaustively detailed in \cite{limpens2019energyscope} and \cite{limpens2024pathway} that are cited several times in the thesis manuscript.  Regarding the uncertainties, the way the ranges are defined and the way they affect the value of some parameters in the model are explained {\color{blue}in Section 1.2 of the core of the manuscript}. Consequently, no modification has been brought to the manuscript regarding this topic.

\subsection{Assumptions - Fine tuning}
\label{fine_tuning}

\begin{mdframed}[style=comment] % Comment from the reviewer
{\color{orange} \textbf{Stefano}}, {\color{purple} \textbf{Sylvain}} \& {\color{teal} \textbf{Stefan}} - Many assumptions, e.g., alpha = 85\% (p. 40) or reward = -300 (Chapter 4). How sensitive are the results to these assumptions? It looks like all these parameters have been tuned to obtain the desired results. How robust are they? Can the methodology be generalized to other models? Can we keep the same values for other applications? You need a model that runs pretty fast, right?

\noindent {\color{teal} \textbf{Stefan}} - Justification of the RL actions selected? why no other actions like the cost of some technologies?

\noindent {\color{purple} \textbf{Sylvain}} - p. 86 - A lot of heuristic and trial and error. Can the methodology be generalized to other models?

\noindent {\color{purple} \textbf{Sylvain}} - Make sure to discuss the hypotheses because one might not agree with all of them. So, it could provide more context but also some limitations and indications of the alternatives (and the spectrum of what is possible).
\end{mdframed}

\noindent I have not carried out analysis to assess the sensitivity of the results to these assumptions. They result either from trial and error or observations made \textit{a posteriori}. However, to discuss these assumptions and the methodological choices, I have added a specific section to each methodological parts of Chapter 1 to provide future users and developers of the methods proposed in this work with guidelines. These aim at helping other researchers in either using the same methodologies and assumptions on their own case study or keep on developing and improving these methodologies.

\begin{mdframed}[style=manuscript]
\textbf{1.1.3 Discussion and guidelines for future researchers} [EnergyScope Pathway]

Given the research questions targeted in this work, EnergyScope Pathway was the right model to choose for several reasons: its low computational burden, its hourly time resolution, its operation and design optimisation of a whole-energy system and the openness of the code and availability of the documentation (see Appendix B.1). 

However, depending on the research questions to address, other pathway energy system models might be more appropriate given their better spatial resolution or higher techno-economical details, for instance. Such models can be used and combined with the other methodological tools presented in this chapter. Although, the main (and only) requirement is to keep a low computational time as getting relevant results from the following approaches usually requires $10^3\sim 10^4$ runs.  In our case, EnergyScope Pathway can run a pathway composed of 7 representative years below 15 minutes on a personal laptop.
\end{mdframed}

\begin{mdframed}[style=manuscript]
\textbf{1.2.4 Discussion and guidelines for future researchers} [Uncertainty quantification]

Carrying out a Global Sensitivity Analysis (GSA) requires two steps: uncertainty characterisation and uncertainty quantification. Characterising the distribution followed by uncertain parameters require a lot of data that are either seldom available or poorly documented in the literature. Consequently, once lower and upper bounds are computed following a method like the one developed by \citet{Moret2017}, a good practice is to consider uniform distributions. Preliminary analyses revealing key parameters impacting the most the feature of interest could direct further researches to refine the characterisation of these parameters.

About the method to quantify the uncertainty, PCE has the advantages to provide Sobol' indices as the impact indicators of the uncertain parameters and extract statistical moments of the output of interest. Based on our experience, considering a second-order PCE is the minimum to get a reasonably low LOO error (below 1\%) and reliable results. However, as presented in Section 1.2.2, starting with a first-order PCE is a good proxy to get a initial insight into the impact of the parameters and discard the negligible ones before moving to a higher order PCE to increase the accuracy of the results.
\end{mdframed}

\begin{mdframed}[style=manuscript]
\textbf{1.3.3 Discussion and guidelines for future researchers} [Agent-based reinforcement learning to support myopic energy transitions]

Reinforcement Learning (RL) was found to be an appropriate approach to explore myopic transition pathways under uncertainties and assess the robustness of policies to support such pathways. As a novice user of a RL framework, we would recommend to opt for SAC as it is sample efficient, ensures a wide exploration and has a low sensitivity to hyper-parameters \cite{haarnoja2018soft}. Using the SAC package developed by \textsc{Stable-Baselines3} allows a handy introduction to apply RL. 

Besides the choice of the algorithm, most of the work when using a RL framework consists in properly define the rules of the games following which the agent interacts with its environment: actions, states and reward. These are defined and discussed in Chapter 4.
\end{mdframed}

\begin{mdframed}[style=manuscript]
\textbf{1.4.3 Discussion and guidelines for future researchers} [Robustness assessment via PCA]

Principal Component Analysis (PCA) was found to be an appropriate method to provide a metric that capture a major part of the system design variance through the transition while keeping the amount of information to a ``graspable'' level of details. Besides selecting the installed capacities of end-use-type categories (electricity, heat, mobility and non-energy), one could discuss the scaling of these data and the management of the outliers. Where scaling the data is fundamental before computing the PCs, future researchers could adapt the scaling factor. To avoid singularities in our PCs, we have decided \textit{a posteriori} to put high and low outliers to the upper and lower fences, respectively. However, before managing the outliers, we would recommend to keep them and assess the provided results in terms of number of PCs and the key technologies that characterise them.

Finally, we have assumed thresholds to limit the number of principal components of each representative year, $\text{PC}_y$ (90\%), the similarity between two of these $\text{PC}_y$ (90\%) and to limit the number of $\text{PC}_{\text{transition}}$ (85\%). These thresholds have been defined based on good practices discussed in the literature and \textit{a posteriori} observations specific to our case study. We would recommend starting with these values and adapting them in a second time if needed to increase or decrease the level of details captured by the PCs.
\end{mdframed}

Moreover, to discuss the rules of the RL games, presented in Section 4.2, I have added {\color{blue}a Section 4.2.4} introduced {\color{blue}at the end of the introductory paragraph of Section 4.2}:

\begin{mdframed}[style=manuscript] % Modification brought to the manuscript
We end this section with justifications of actions, reward and states implemented in this work as well as guidelines for researchers that would like to apply the RL approach on their own case study with their own model.
\end{mdframed}

\begin{mdframed}[style=manuscript] % Modification brought to the manuscript
\textbf{4.2.4 Discussion and guidelines for future researchers}

As introduced in Chapter 1, most of the work in applying RL is the definition of the interactions between the agent and its environment (\ie actions, reward and states) that are very dependent on the case study and the research questions to answer. The elements presented in this work result from several trial and errors to end up with meaningful results according to our research questions.

Besides mimicking potential actual policies, the actions chosen in this work have a direct translation into constraints and we can therefore assess their effectiveness through the fact they are binding or not. In this work, we have investigated other actions like incentivising solar PV panels and wind turbines by ``artificially'' reducing their CAPEX. The result was not conclusive as these technologies must take part to the Belgian energy transition and because it was harder to assess the impact of this action. 

The reward function was designed to first aiming at respecting the \ce{CO2} budget then minimising the total transition cost. The -300 penalty given in case of an infeasible optimisation problem was arbitrarily set. \textit{A posteriori}, it seems to be a well-defined penalty given its significant relative difference with the values taken by the reward otherwise, between -120 and 44 (see Figure 4.4). Besides this penalty and given the observed results, we recommend to start with the same reward function if the objective is similar, first target cumulative emissions then cumulative costs. This requires to define the \ce{CO2} budget according to a certain sharing principle (see Section 2.5) and to compute reference total transition cost. However, other researches focusing on reaching carbon neutrality by 2050 could define a binary reward function as +1 for reaching the objective and -1 otherwise.

Finally, states aim at representing the information relevant to the agent to efficiently learn and progress through the transitions. For this reason, on top of reward-related features (cumulative emissions and costs), we added other indicators that are actually monitored to help decision-makers assessing their policies to reach their targets (share of renewables in the mix and the system overall efficiency). For other studies, one might consider other information like the metrics considered by \citet{pickering2022diversity} (\eg heat electrification, average national import or level of curtailment).

In conclusion, future studies might start from the actions-reward-states defined in this work and adapt these rules depending on the research questions to answer, the case study and the energy system optimisation model.
\end{mdframed}

Finally, I have added a sentence {\color{blue}in the conclusion of the manuscript} to remind that the methods and the assumptions presented in this work can serve as a good basis but will need to be adapted depending on the model and the case study:

\begin{mdframed}[style=manuscript] % Modification brought to the manuscript
Besides the formulated recommendations, the assumptions and the formulation choices made in this work can serve as a basis for further researches but will need to be adapted according to the model that is used or the case study that is studied.
\end{mdframed}

\subsection{Graphs and results}
\label{general_graphs_results}

\begin{mdframed}[style=comment] % Comment from the reviewer
{\color{orange} \textbf{Stefano}}, {\color{teal} \textbf{Stefan}} \& {\color{purple} \textbf{Sylvain}} - Many axes labels are missing, and the captions are often insufficient to read the figures. 
\end{mdframed}

\noindent I agree and have gone through all the figures of the manuscript to clarify them by adding axes labels or extending the caption where it was needed. Figures where y-axis (and its label) is missing are distribution (or density) plots (\ie Figures 3.6, 4.4, 4.5, 4.6, 4.8 and E.1).  Consequently, similarly to an histogram, the feature of interest and its span of values are represented on the x-axis. The y-axis has no real interest in terms of information. What matters is the shape of the curve to be able to point out for which values/ranges of values, the feature of interest occurred more or less often.  For this reason, I have not added y-axis for the aforementioned figures but rather improved their respective caption, emphasising, among other things, that these figures represent distribution of occurrences. For instance, the caption of {\color{blue}Figure 4.4} is now more explicit:

\begin{mdframed}[style=manuscript] % Comment from the reviewer
Exploration of the state space over the learning process: distribution of occurrence of cumulative emissions (left) and costs (right). Number of remaining attempts decrease with time since infeasible problem and solutions overshooting the \ce{CO2} budget are discarded prematurely, \ie before reaching 2050. Besides infeasible problems, distributions labelled as ``Failure'' represent the attempts that overshot the \ce{CO2} budget by 2050 at the latest. The majority of successful transitions have cumulative emissions much lower than the \ce{CO2} budget and are cheaper than the REF case. 
\end{mdframed}

\begin{mdframed}[style=comment] % Comment from the reviewer
{\color{teal} \textbf{Stefan}} - Some big topics are missing like storage. Why is storage not more discussed? 
\end{mdframed}

\noindent As discussed during the private defense, storage is not more discussed in the manuscript for one main reason: the lack of variation through the pathway and under uncertainties. The storage technologies, for the case of Belgium assuming a copper-plate approach, consist in two main technologies: methane storage mostly (about 15\,TWh) and, to a smaller extent, seasonal thermal storage (about 3\,TWh), with limited variations through the transition (see Figure \ref{fig:STORAGE_2}). Even though there is no limitation on the yearly quantity of methane (fossil or e-methane), for each representative year, the model is constrained to import the same quantity for each hour of each typical day. For this reason, the model decides to install methane storage that is emptied during the winter and refilled during the summer, as methane is mostly used to produce heat. Analyses under uncertainties have not shown major variations either. Moreover, storage assets are intrinsically linked to the conversion technologies and, consequently, can be seen as by-products of the optimisation results. For these reasons, I have rather focused my analyses on the use of resources and the installation and operation of conversion technologies. Given the discussion during the private defense and these further explanations, I have not brought modification to the manuscript in this regard.

\begin{figure}[htbp!]
\centering
\includegraphics[width=0.7\textwidth]{STORAGE_2.pdf}
\caption{Storage installed capacities for the REF case}
\label{fig:STORAGE_2}
\end{figure}

\begin{mdframed}[style=comment] % Comment from the reviewer
{\color{teal} \textbf{Stefan}} - Lot of the results hinge on the import of e-fuels with the assumption that they have GWP = 0; is it justified? Those fuels are directly available knowing they are not.
\end{mdframed}

\noindent To succeed its energy transition, Belgium will need radical changes, ``unicorn'' solutions, and one of them would be the import of electrofuels. As presented in the introduction and reminded in the conclusion of the manuscript: ``Chemically similar to their fossil-based equivalents, they [the electrofuels] can benefit from the current infrastructures and provide seasonal storage and flexibility against the intrinsic intermittency of solar and wind electricity.'' Besides these intrinsic advantages, considering these fuels as infinitely available (like their fossil equivalents) and with 0-GWP allow finding a solution for the Belgian case study, without considering a decrease of the demands.  These are strong assumptions that can be challenged as suggested in the conclusion: ``A future work would be to constrain the deployment of local renewables or the availability of electrofuels. Combined with a reduced end-use-demand, this could provide different pathways where electrofuels would be produced more locally using carbon capture technologies. This could also show the limits of the Belgian energy system to meet such an ambitious CO2-budget in its energy transition when solely relying on technological progress, highlighting the need for demand reduction. ''

The assumption about 0-GWP molecules is similar to the one considered in \cite{PATHS2050} which is the most recent study that addressed the case of the Belgian energy transition. 

As discussed during the private defense, the first key-message of this work is the need for this (mix of) ``unicorn'' solution(s) given the ambitious objectives set to mitigate climate change.  The need for this radical change and the fact that solutions rely a lot on assumed to be carbon-neutral and directly available electrofuels are explicitly mentioned {\color{blue}in the conclusions of Chapter 3}: 

\begin{mdframed}[style=manuscript] % Modification brought to the manuscript 
Given the ambitious \ce{CO2} budget (\ie 30-year budget representing 10 years of the current emissions), the global sensitivity analysis highlights the need for radical changes. Assumed to be carbon-neutral and directly available by a large amount, one of the ``unicorn'' solution for the case of Belgium is the import of renewable electrofuels, even at early stages of the transition. This is why the most impacting parameter on the variability of the total transition cost (around 45\%) is the cost of purchasing these energy carriers.
\end{mdframed}

and reminded {\color{blue}in the conclusions of the manuscript}:

\begin{mdframed}[style=manuscript] % Modification brought to the manuscript 
Besides a wider electrification of the system and the integration of more efficient technologies, Belgium needs to go for radical changes, ``unicorn'' solutions. One robust policy would consist in massively investing in importing electrofuels.
\end{mdframed}

\subsection{Structure}
\label{structure}

\begin{mdframed}[style=comment] % Comment from the reviewer
{\color{orange} \textbf{Stefano}} \& {\color{teal} \textbf{Stefan}} - The thesis structure makes it quite difficult to follow. Having all the methodology summarised in Chapter 1, makes it difficult to link it to the different chapters. As changing the structure may be now a major work, I would invite you to think how to better connect the different parts. 
\end{mdframed}

\noindent Initially, I wanted to group methodology and results per kind of analysis, \ie UQ on Pathway (Chapter 3), RL (Chapter 4) and PCA (Chapter 5) but that would have made the RL, and even more PCA, methodology parts come too far in the manuscript. For this reason, I made the choice to group all the methodological approaches together in Chapter 1 and the results in their respective chapters.  Given this explanation, I have added one sentence {\color{blue}at the end of the introductory paragraph of the RL method in Chapter 1}:

\begin{mdframed}[style=manuscript] % Modification to manuscript
As Chapter 2 presents the input of the case study and Chapter 3 details the results of the GSA on the pathway model, the reader interested by the results of the RL method is invited to go to Chapter 4.
\end{mdframed}

\noindent and {\color{blue}at the end of the introductory paragraph of the PCA-based method in Chapter 1}:

\begin{mdframed}[style=manuscript] % Modification to manuscript
The reader interested by the results of the PCA-based method is invited to go to Chapter 5.
\end{mdframed}

\noindent Finally, I have added a schematic at the end of the Introduction of the thesis to clarify the structure and highlight that Chapters 4 and 5 can be read after the methodological sections 1.3 and 1.4, respectively (see Figure \ref{fig:intro:Thesis_Structure}).

\begin{figure}[htbp!]
\centering
\includegraphics[width=0.7\textwidth]{Thesis_Structure.pdf}
\caption{Structure of the thesis. Beyond Chapter 2 that describes the case study of Belgium, Chapters 3, 4 and 5 collect the analyses resulting from the application of the different methodologies developed in Chapter 1. Given the (quasi-)independence of the analyses presented in the last three chapters, they can be read separately.}
\label{fig:intro:Thesis_Structure}
\end{figure}


\begin{mdframed}[style=comment] % Comment from the reviewer
{\color{violet} \textbf{Christophe}} - Formal description of the problem is missing even if described in the text to make the link between the ideas and the implementation. Add some flow charts to introduce key elements
\end{mdframed}

\noindent Besides the mathematical details brought in the description of the optimisation problem, this point is addressed with the structure chart presented in the previous comment as well as the improvement of Figure 1.4 presenting the implementation of the myopic approach as well as the way uncertainties and RL change the values of some parameters (see Figure \ref{fig:MY_process_code} of the present document).

\subsection{References}
\label{references}

\begin{mdframed}[style=comment] % Comment from the reviewer
{\color{violet} \textbf{Christophe}} - State of the art mainly European. Why not reference to other regions of the world.
\end{mdframed}

\noindent
It is an interesting topic for further investigations, especially how they address the subject in other parts of the world like Australia/Asia. This could be linked with the political perception of energy system optimisation models and the subsequent analyses. However, no modification has been brought to the manuscript regarding this topic.

\subsection{Documentation}
\label{documentation}

\begin{mdframed}[style=comment] % Comment from the reviewer
{\color{purple} \textbf{Sylvain}} - Where is the repository for the code? Don't forget the documentation?
\end{mdframed}

\noindent As discussed during the private defense, this is a task that is already planned for the year to come, after the public defense. A better reference towards these repository and documentation will be given in the subsequent paper on RL. Consequently, there has not been further work done in this regard in the manuscript by the public defense.

\section{Introduction}
\label{Introduction}

\begin{mdframed}[style=comment] % Comment from the reviewer
{\color{orange} \textbf{Stefano}} \& {\color{teal} \textbf{Stefan}} - At page 1, it is mentioned that you focus on the technical levers of the transition, ``renewables'' and ``efficiency''. From this I understand that you are not focusing on ``sufficiency''. This seems to contradict a statement at p. 2, where it is mentioned that one objective of the thesis is to ``support interdisciplinary projects in the assessment of sufficiency policy''. I suggest clarifying this potential misunderstanding.  
\end{mdframed}

\noindent I totally agree with this remark, especially because I do not want to overlook the necessary third pillar of the transition which is ``sufficiency'', even though my thesis does not address directly this aspect. Consequently, I have adapted {\color{blue} page 2 of the introduction}:

\begin{mdframed}[style=manuscript] % Modification brought to the manuscript
Among all the lenses through which it is necessary to assess sufficiency policies, one of the objectives of this work is to support these interdisciplinary projects by providing informed techno-economic guidelines.
\end{mdframed}

\noindent This technical contribution to the interdisciplinary debate of sufficiency is reminded in {\color{blue} the last sentence of the conclusion}.

\begin{mdframed}[style=manuscript] % Modification brought to the manuscript
In this sense, this work provided insight about possible transition pathways for Belgium to bring the technical dimension into the intrinsically political and interdisciplinary discussions and decisions that must be made in the coming years.
\end{mdframed}

\section{Chapter 1 - Methodology}
\label{methodo}

\subsection{EnergyScope Pathway}
\label{ESPathway}

\subsubsection{End-of-time-horizon, myopic pathway and salvage value}

\begin{mdframed}[style=comment] % Comment from the reviewer
{\color{orange} \textbf{Stefano}} - Often, multi-stage models are affected by the end-of-time-horizon effects. Did you observe those and how did you deal with them? See, e.g. Figure 3.3 p. 66.
\end{mdframed}

 \noindent To cope with these ``end-of-time-horizon effects'', we have implemented the salvage value, $\textbf{C\textsubscript{inv,return}}$. This represents the residual investment that would remain after 2050. As detailed in Section 1.1.1: ``the salvage value avoids penalising the capital intensive technologies towards the end of the transition'' \cite{poncelet2016myopic}. The sensitivity analysis on the formulation of the salvage value has been added in Appendix B.2.2 (see further comments on the salvage value).

As discussed during the private defense, instead of implementing a salvage value another alternative would be to implement artificial representative years after 2050 that are then excluded from the total transition cost and emissions in while post-processing the results of the optimisation. Even though the choice of the formulation of the salvage value accounts for assumptions (discussed in Appendix B.2.2), we opted for this rather than ``artificial future years'' for several reasons. First of all, it allows mitigating the computational burden of the pathway problem. Indeed, the more there are representative years, the bigger are the matrices and the longer it takes to solve the problem. Second, the salvage value-approach is also used in other works \cite{poncelet2016myopic,prina2019transition}. Finally, implementing ``artificial  future years'' would require to either collect more data, which is quite time-consuming, or make other kind of assumptions on the evolution of costs, GWP and other features beyond 2050. For instance, we could artificially replicate the year 2050 and reach a steady state.

Since the implementation of the salvage value and the formulation choices are the work developed by \citet{limpens2024pathway} (partly integrated in Appendix B) and given the above explanations, I have not brought further modification to the manuscript in this regard.

\begin{mdframed}[style=comment] % Comment from the reviewer
{\color{teal} \textbf{Stefan}} - With the myopic approach, are you predicting what will come or drawing a path that is the best? Is that a desirable feature to mimic the way we proceed now. Why artificially limit the model as we know we have to go to zero in 2050?
\end{mdframed}

\noindent As written in the manuscript and discussed during the private defense, the purpose of the myopic approach is to challenge the assumption of the complete knowledge about the uncertain future and mimic the short-sightedness of the decision-making process. This sequential optimisation allows showcasing robust policies to support myopic transitions (see Chapter 4) and the robustness of roadmaps (see Chapter 5). The myopic approach is also the methodological basis to support the implementation of the RL and PCA-based methods. 

Given these explanations and what is already in the manuscript, I have not brought modification in this regard.

\begin{mdframed}[style=comment] % Comment from the reviewer
{\color{teal} \textbf{Stefan}} - p.18 : Surprising that myopic leads to a sooner investment to the transition? Is it an effect of the salvage value? was there a sensitivity analysis on this salvage value? Perhaps include more details on this. The salvage value plays a large role in the decision of the model. Give more details to give more confidence in the choice.
\end{mdframed}

\noindent Indeed, it is an effect of the salvage value as already written in the last paragraph ``Impact of myopic formulation on the system'' of Section 1.1: ``The main difference lies in the myopic transition itself and especially in the earlier
deployment of PVs and offshore wind turbines. These induce the reinforcement of the grid that is a capital-intensive and long-lifetime asset. This is mostly due to impact of the salvage value, Equation (1.4), in the objective function.''.

Sensitivity analysis has been performed on the formulation of the salvage value by \citet{goffauxpathway}. The overall conclusion of this sensitivity analysis was that the salvage value expression has little influence on the energy system at the end of the time horizon. The methodology and results of this analysis is now summed up {\color{blue}in Appendix B.2} 

\begin{mdframed}[style=manuscript] % Modification brought to the manuscript
In the work of \citet{goffauxpathway}, we carried out a sensitivity analysis on the formulation of the salvage value. First and foremost, this analysis confirmed the need to account for the salvage value to avoid penalising the capital intensive technologies towards the end of the transition \cite{poncelet2016myopic}. Then, the sensitivity analysis focused on two elements in the expression of the salvage value: $\tau_{phase}(p)$ and $\textbf{F\textsubscript{decom}}(p2,p,i)$. 

In Equation (B.22), $\tau_{phase}$ is the annualisation factor corresponding to the phase where the technology was built. With this expression, if a technology is still operational for half of its lifetime after 2050, then half of its initial investment is subtracted to the total transition cost. However, one could consider the depreciation of the asset where a technology that is still operational for half of its lifetime is worth less than half of its initial investment cost. 

On top of this, we also investigated the impact of accounting for the decommissioned technologies or not in the expression of the salvage value.  In Equation (B.22), if a technology is decommissioned before 2050, then the salvage value of this technology will be 0. This is relevant since a technology decommissioned before 2050 would not be available after 2050. The issue is that the model could keep unnecessary technologies to subtract their salvage value from the total transition cost in order to decrease it. This would be done if the fixed OPEX of an unused technology is lower than its salvage value. Therefore, one could wonder if it would not be better to take into account the decommissioned technologies in the salvage value. In that case, a technology that would have been operational after 2050 but that had been prematurely decommissioned before 2050, would have a non-0 salvage value.

In a nutshell, the overall conclusion of this sensitivity analysis was that considering a salvage value was fundamental in our pathway formulation. However, the expression of the salvage value itself has little influence on the energy system at the end of the time horizon.  The interested reader is invited to refer to the work of \citet{goffauxpathway} for the numerical results of this analysis.
\end{mdframed}

\noindent and a reference to this appendix is present {\color{blue}in Section 1.1 where Equation (1.4) is detailed}:

\begin{mdframed}[style=manuscript] % Modification brought to the manuscript
After assessing the sensitivity of the formulation of the salvage value (see Appendix B.2), we have kept the one given by Equation (1.4), being also similar to the expression used by \citet{prina2019transition}.
\end{mdframed}

\subsubsection{Discount rate and annualisation}

\begin{mdframed}[style=comment] % Comment from the reviewer
{\color{purple} \textbf{Sylvain}} - Wouldn't it be more appropriate to talk about ``discount rate'' rather than ``interest rate''?
\end{mdframed}

\noindent Indeed, as we want to capture the time preference associated with obtaining finance for the project, effectively quantifying the trade-off between present and future financial value, it is more appropriate to refer to ``discount rate'' when considering the parameter $i_{\text{rate}}$. EnergyScope considers a state that needs to decide when investing in which technology. There is no consideration of return on investment as it would be the case for private investors. We have changed every occurrence of ``interest rate'' into ``discount rate'' {\color{blue}in the entire manuscript}.

\begin{mdframed}[style=comment] % Comment from the reviewer
{\color{purple} \textbf{Sylvain}} - Value of the discount rate is generally above 6\% in other works. You should have a better justification of selecting 1.5\%.
\end{mdframed}

\noindent Unlike snapshot models where the discount rate aims at annualizing investments, discounting in pathway models aims at evaluating the value of future expenses compared to present expenses. In other words, the higher the discount rate, the lower the impact of future extra costs. In their assessment of the role of the discount rates in energy systems optimisation models, \citet{garcia2016role} showed that higher ``social discount rate'', affecting the whole-energy system, including the operation (in opposition to technology-specific ``hurdle rates''), favours fossil-based technologies and postpones investments in more capital-intensive assets like solar PV and wind turbines. Where this is visible in the perfect foresight (PF) approach, the myopic optimisation of transition pathways is less affected by this given the shorter time-horizon considered in the sequential optimisations, 10 years versus 30 years in the PF approach. Even though the discount rate is usually between 7.5\% and 12\% in other studies, we have set its nominal value in line with G. Limpens' work \cite{limpens2021generating}. I have added some explanations in this regard {\color{blue}in Section 1.1.1}:

%\begin{figure}[htbp!]
%\centering
%\includegraphics[width=10cm]{tau_phase.png}
%\caption{$\tau_{phase}$ versus the timing of the expenses (columns) and the discount rate, $i_{rate}$ (rows).}
%\label{fig:tau_phase}
%\end{figure}

\begin{mdframed}[style=manuscript] % Modification brought to the manuscript 
In EnergyScope Pathway, $\tau\textsubscript{\emph{phase}}$ aims at depreciating the value of expenses done in the future. The discount rate accounted for in this annualisation factor, $i_{\text{rate}}$, is considered as identical for all the technologies of the system. In line with \citet{limpens2021generating}, we set the nominal value to $i_{\text{rate}}=1.5\%$ which is lower than values taken in other studies, between 7.5\% and 12\% \cite{meinke2017energy,simoes2013jrc,EuropeanCommission2016}. In the 30-year time window of the perfect foresight approach, having a higher value of discount rate would postpone investments in more capital-intensive assets. However, in a myopic vision, this effect is more limited given the shorter time window. 
\end{mdframed}

\begin{mdframed}[style=comment] % Comment from the reviewer
{\color{orange} \textbf{Stefano}} - Can you explain the logic of the annualization factor in the pathway model?

\noindent {\color{purple} \textbf{Sylvain}} - p.13: Why no annualizing over the lifetime of the unit? Would that lead to different results? 
\end{mdframed}

\noindent Accounting for the depreciation of money with time (see previous comment), we considered that the assets are fully paid at the moment they are installed. For this reason, we consider the arithmetic average between the investment cost of the assets at the year starting the phase of installation and the year ending it. This choice was done for two main reasons: potential decommissioning and lack of data beyond 2050. In the Pathway version of EnergyScope, the asset will be used over its whole lifetime unless it is prematurely decommissioned. Therefore, annualising over the entire lifetime of the unit might induce a bias in case of this premature decommissioning. Then, for assets with longer lifetime and/or installed later in the transition, annualising over the lifetime would require to have data beyond 2050.

Regarding the impact on the results, my educated guess is that this impact would be limited as the technologies of a same sector are assumed to have similar trends in terms of the evolution with time of their respective investment costs. Consequently, the major impact would be to end up with a lower total transition costs but the design of the system would be marginally impacted.

Given these further explanations, I have not brought any modification to manuscript in this regard.

\begin{mdframed}[style=comment] % Comment from the reviewer
{\color{purple} \textbf{Sylvain}} - p.71 - Differentiated discount rates could be useful.
\end{mdframed}

\noindent As written in Chapter 3: ``In practice, the discount rate would vary depending on the technology investment risk''. Depending on the technology readiness level, private investors would be more or less risk-averse. However, EnergyScope only considers a vision of a central-planner where a single agent makes the investment decisions without differentiating the sources of the capital, \ie private and public. Assuming a differentiated discount rate would require to make assumptions about the private-public distribution of the capital. These further explanations for keeping a single value of discount rate have been added {\color{blue} after Eq. (1.3) when discussing the annualised phase factor}.

\begin{mdframed}[style=manuscript] % Modification brought to the manuscript 
The discount rate accounted for in the annualisation factor is considered as identical for all the technologies of the system. In practice, the discount rate would vary depending on the technology investment risk. Depending on the technology readiness level, private investors would be more or less risk-averse. However, EnergyScope only considers the vision of a central-planner where a single agent makes the investment decisions without differentiating the sources of the capital, \ie private and public. Assuming a differentiated discount rate would require to make assumptions about the private-public distribution of the capital. For this reason, we have decided to keep an identical value of discount rate for all the technologies.
\end{mdframed}

\begin{mdframed}[style=comment] % Comment from the reviewer
{\color{teal} \textbf{Stefan}} - Interest rate must be revisited if a paper is published.
\end{mdframed}

\noindent See previous comments.

\subsubsection{Power grid and transport infrastructure}

\begin{mdframed}[style=comment] % Comment from the reviewer
{\color{purple} \textbf{Sylvain}} - How do you model the power grid interconnections with the neighbouring countries as well as the transport network of the different energy carriers within the country? p.63 - The model decides to use ammonia instead of SNG $\rightarrow$ is the transport infrastructure accounted for? p.73 - How is the transport infrastructure accounted for?
\end{mdframed}

\noindent Regarding the power grid interconnections with the neighbouring countries, similarly to the import of other energy carriers, there is no representation of those physical routes. The whole-energy system of the country is represented as a single-node and the imported energy commodities are available ``at the front door'' of the system. It goes the same way for the infrastructures to transport energy carriers within the system. As it is represented assuming a copper-plate approach, all the demands, resources and conversion technologies are concentrated in one single node. This is now explicitly written {\color{blue}in a new paragraph of Section 2.2}:

\begin{mdframed}[style=manuscript] % Modification brought to the manuscript 
Regarding the power grid interconnections with the neighbouring countries, similarly to the import of other energy carriers, there is no representation of those physical routes. The imported energy commodities are available ``at the front door'' of the system.  In terms of spatial resolution of the country itself, it is modelled as a single-node. This is similar to the copper plate model of a power grid. In other words, the infrastructures to transport the energy carriers within the country are not considered.  It is assumed that the demands have to be supplied by the production, without considering the flows between the producers and the consumers. Yet, adapting the networks is accounted for in terms of the required investments. For instance, a high share of VRES requires an investment to reinforce the power grid (\ie 368\,M€/GW of additional installed capacity of VRES). 
\end{mdframed}

\subsection{Uncertainty quantification}
\label{methodo_UQ}

\begin{mdframed}[style=comment] % Comment from the reviewer
{\color{orange} \textbf{Stefano}} - p. 23: is PCE an appropriate method for LP / MILP models. Why is PCE needed if EnergyScope is already very fast? There is a good fit (1\% LOO error) for the total cost, which is often quite linear, but how about technology choice / sizing?
\end{mdframed}

\noindent Like in Diederik Coppitters' works \cite{coppitters2021robust,coppittersthesis}, the ``ultimate'' objective of PCE is to build up a surrogate model, $\hat{M}$, to serve a robust design optimisation (RDO) process. For an original non-linear model $M$, this surrogate model, formed of as a series of polynomials,  has the main advantage to run much faster than the original model. In my case, I used PCE as an uncertainty quantification (UQ) tool without going up to the RDO step. As written {\color{blue}in the first paragraph of Section 1.2.2}, I (only) aim at getting the statistical moments of the quantity of interest and determine Sobol' indices. Similarly, in Chapter 5, I used PCA to extract main direction of variation when it comes to the design of the system without performing runs using these PCs that are another representation of the original model. Consequently, regarding this objective, PCE is an appropriate method for EnergyScope Pathway. 

\noindent When the LOO error is higher than 1\% (but limited to 20-25\%), which is the case when performing the PCE on technology size/choice (\ie installed capacity of SMR by 2050) or the import of renewable molecules (see Section 3.2.2), the approach remains valid to qualitatively rank the parameters, {\color{blue}as discussed in the paragraph ``Qualitative ranking based on Sobol' indices'' of Section 1.2.2}.

\noindent Given these explanations and the elements already present in the text, I have not brought further modifications in the manuscript.

\begin{mdframed}[style=comment] % Comment from the reviewer
{\color{purple} \textbf{Sylvain}} - p. 57: Why a uniform distribution? A gaussian would clearly be more appropriate. Maybe some uncertainties are underestimated (because clipped by uniform distribution).
\end{mdframed}

\noindent As already written in the first paragraph of Section 1.2.1, given the scarcity of data to actually define a distribution of uncertainty for the parameters, the common practice is to consider uniform distribution. As detailed by \citet{coppittersthesis}: ``Distributions are typically defined based on large datasets, which are not always at hand for each parameter. Therefore, when no meaningful information can be extracted on the distribution from a limited dataset, a uniform distribution is typically assumed, which assigns an equal probability to each value within a range. This approach is similar to assigning an interval to a parameter, but takes advantage of the central tendency when uniform distributions are propagated through a system model (\ie central limit theorem)''. Given what is already in the thesis manuscript and these further explanation, no modification has been brought to the manuscript in this regard. 

\subsection{Principal Component Analysis}
\label{methodo_PCA}

\begin{mdframed}[style=comment] % Comment from the reviewer
{\color{orange} \textbf{Stefano}} - p. 42: Figure 1.17. What is your definition of robustness? Having a tight distribution but all shifted towards very suboptimal costs seems to fit your definition. Please clarify.
\end{mdframed}

\noindent As discussed during the private defense, it is more the spread of variation of the distribution that matters to characterise the robustness rather than the average position. This is now clarified {\color{blue}at the end of Section 1.4.2}:

\begin{mdframed}[style=manuscript] % Modification brought to the manuscript
It is really the range over which the distribution spans, $2\cdot \mathrm{MOE}$, that matters to characterise the variability of the projections of a roadmap. Indeed, as $\text{PC}_{\text{transition}}$ are vectors composed of positive and negative coefficients, the average position of the distribution, $\mu$, does not bring insight about the robustness of a roadmap.
\end{mdframed}

\begin{mdframed}[style=comment] % Comment from the reviewer
{\color{purple} \textbf{Sylvain}} - p. 39 Why are their outliers in the first case? Isn't a bit arbitrary to set them to a min/max value?
\end{mdframed}

\noindent As detailed in Section 1.4.2, even though these points result from the optimisation and make sense mathematically, they are defined as outliers as they fall out of the whiskers of their respective box plot. The way we managed these outliers is arbitrary but is a fair trade-off between leaving them as they are which would mislead the direction of PCs and completely removing them. As extensively discussed in Section 5.1.1, this pre-processing step will mostly affect the LT-heat sector where heat pumps running on methane are installed in rare cases, \ie less than 10\% of the 1260 transitions. Given the size of this sector, \ie between 22\% and 27\% of the EUD for Belgium, and the LP approach where there is a complete substitution of a technology by another one under certain conditions, these gas-HP can have a major weight in the system. However, to keep the main trends of variation without being ``polluted'' by outlying results (here, less than 10\% of the cases), we proceeded that way.

Given what is already in the text and these explanations, I have not brought modification to the manuscript in this regard.

\section{Chapter 2 - Case study}
\label{case_study}

\subsection{General and introduction}
\label{methodo_general}

\begin{mdframed}[style=comment] % Comment from the reviewer
{\color{orange} \textbf{Stefano}} - I am a bit underwhelmed by the many nitty-gritty data in this Chapter. While it is surely important to document everything, couldn’t at least part of this be moved to the Appendix?
\end{mdframed}

\noindent Most of the data about the Belgian case study have been collected by \citet{limpens2021generating}. The objective of this Chapter was to focus on my contributions: non-energy demand (NED), electrofuels, SMR, uncertain ranges for the pathway model and \ce{CO2} budget. However, to put these contributions in perspective with similar features in the model, I have decided to keep general information about the end-use demands, the resources and the conversion technologies. Consequently, I have kept the same level of information in Chapter 2 rather than in appendix.

\subsection{Conversion technologies}
\label{methodo_technologies}

\begin{mdframed}[style=comment] % Comment from the reviewer
{\color{purple} \textbf{Sylvain}} - p. 51 Why investigating SMR?
\end{mdframed}

\noindent As discussed during the private defense, we have decided to investigate to role of nuclear energy in the future for two main reasons: hot topic in Belgium and similar study by EnergyVille \cite{PATHS2050}. First, when presenting the results of our work at conferences/meetings, we have been asked many times how our results, massively relying on imports of renewable molecules, were influenced by the installation of nuclear power plants by 2050. Second, during my thesis, EnergyVille delivered a report PATHS2050 where, among others, they investigated the impact of SMR. As they observed that installing SMR would significantly reduce the need to import renewable molecules, we also wanted to assess this impact in our work. Given these explanations, no further modification has been brought to the manuscript in this regard.

\begin{mdframed}[style=comment] % Comment from the reviewer
{\color{purple} \textbf{Sylvain}} - p. 53 4850€/kW is at the lower end of the EnergyVille scenario for SMR (4500€/kW). This is a very optimistic price forecast! Current tenders for 3rd generation are higher than 100€/MWh. There are papers showing that SMRs are not really expected to be cheaper than 3rd generation. How realistic is 41 €/MWh? LCOE of SMRs would be much higher if: - the capital cost was more realistic and, - the Capacity factor was lowered to account for their flexible use.
\end{mdframed}

\noindent As discussed during the private defense, similarly to EnergyVille \cite{PATHS2050}, we have considered the same CAPEX for SMR as the one considered for conventional nuclear power plants. Where EnergyVille considers a nominal CAPEX of 7500€/kW (like the current large Gen III design of Hinkley Point in the UK), we considered 4850€/kW like the CAPEX of conventional nuclear units in EnergyScope. 

On top of the uncertainty applied to the CAPEX of SMR [-40\%, 44\%] considered in the thesis, I carried out a test on these assumptions by plotting the LCOE of electricity-production units considering different assumptions for SMR (see Figure \ref{fig:LCOE_comparison}). Even considering 50\% more expensive SMR subject to a higher discount rate (5\% versus 1.5\% for the other technologies), we notice that SMR keeps on being more cost-competitive than other flexible production units, like e-ammonia CCGTs, that are substituted by SMR when the latter can be installed (see Chapter 3). In other words, it means that the optimistic values assumed for SMR in the thesis lead to a cheaper whole-energy system but do not affect its design, \ie the installed technologies.  Consequently, no further modification has been brought to the thesis regarding this comment.

\begin{figure}[htbp!]
\centering
\includegraphics[width=0.49\textwidth]{LCOE_line_2.pdf}
\includegraphics[width=0.49\textwidth]{LCOE_line_3.pdf}
\label{fig:LCOE_comparison}
\caption{LCOE per technology producing exclusively electricity considered in the thesis (left) and same LCOE but, this time, with a higher CAPEX (7270€/kW versus 4850€/kW in the thesis and a discount rate of 5\% on this technology. Even considering these less favourable assumptions for SMR, it still outcompetes other flexible production units like CCGT running on e-ammonia.}
\end{figure}

\begin{mdframed}[style=comment] % Comment from the reviewer
{\color{teal} \textbf{Stefan}} - SMR must be revisited if a paper is published.
\end{mdframed}

\noindent See previous comments.

\section{Chapter 3 - Atom-vs-molecules}
\label{Chap_atom_vs_molecules}

\subsection{General}

\begin{mdframed}[style=comment] % Comment from the reviewer
{\color{orange} \textbf{Stefano}} - Would it be fair to say that the scope of this Chapter is adding one technology (SMR) to the model? Is this a significant contribution?
\end{mdframed}

\noindent Besides having added SMR to the model which is one of the contributions regarding the case study (Chapter 2), the main contributions of this chapter is the global sensitivity analysis (GSA) performed on the optimisation of the transition pathway of a whole-energy system with 34 uncertain parameters. When applied to the case study of Belgium, this analysis also allowed to highlight the competition between SMR and some of the imported electrofuels (mostly e-ammonia and e-methane) by 2050. 

To make it clearer from the beginning of the Chapter, I have extended {\color{blue}the contributions of Chapter 3}:

\begin{mdframed}[style=manuscript] % Modification brought to the manuscript
This GSA applied to a model optimising the transition pathway of a whole-energy system with these many uncertain parameters is the main contribution of this chapter. Through this analysis, we also managed to point out the impact of integrating SMR in the Belgian energy system as well as the main drivers of the import of e-hydrogen, e-methane, e-ammonia and e-methanol by 2050.
\end{mdframed}

\begin{mdframed}[style=comment] % Comment from the reviewer
{\color{orange} \textbf{Stefano}} - What is the learning of this chapter? I was underwhelmed by the long list of results. What is the take-away?
\end{mdframed}

\noindent The results are deeply detailed in Sections 3.1 and 3.2. These sections address the impact of adding SMR in the deterministic solution (\ie uncertain parameters at their respective nominal values) of the Belgian energy transition and global sensitivity analyses on different outputs of interest, respectively.  The purpose of these two sections is to give numerical substance to the main conclusions and take-aways that are listed in Section 3.3:
\begin{itemize}
\item If available, SMR is installed and mostly affect the electricity and high-temperature sectors;
\item To minimise the variation of the total transition cost, the key parameter is the cost of purchasing electrofuels where parameters related to SMR (availability and CAPEX) have much lower impact;
\item E-ammonia and, to a lesser extent, e-methane imports by 2050 are impacted by the availability of SMR whereas e-hydrogen and e-methanol are rather influenced by variation related to transport technologies and non-energy demand, respectively;
\item Even though SMR and electrofuels are in competition, they do not limit the deployment of VRES in Belgium. In other words, investing in wind and solar is a must-do for the Belgian energy transition;
\item Given the set of assumptions, especially considering the end-use-demands and the local potential of VRES, Belgium will need to import molecules in the near future. On top of investing into VRES, it seems reasonable to invest in the imports of these electrofuels as betting on SMR means letting the short-term emissions go up and, potentially, requiring significant (and quick) adjustments in case the technology is not ready on time to still respect the \ce{CO2} budget.
\end{itemize}

Given these explanations and what is already written in the text, I have not brought further modification to the manuscript in this regard.

\subsection{Graphs}
\begin{mdframed}[style=comment] % Comment from the reviewer
{\color{orange} \textbf{Stefano}} - p. 74 - Figure 3.8: this graphs looks potentially interesting but it is not sufficiently well explained.
\end{mdframed}

\noindent The objective of this set of figures is to point out the key parameters that drive the variation of some outputs of interest. Taking the case of Figure 3.8, we focus on the import of e-methanol in 2050. On the x-axis, we represent the value taken by the uncertain parameters in the different samples. As the range is not the same for all the parameters, we have limited this axis from ``Min'' to ``Max''. The y-axis represents the feature of interest. The drawn lines represent average trend lines between the output of interest and one specific parameter. For the case of Industry EUD, for samples of the GSA where this uncertain parameter is at 35\% of its range, the mean value of methanol imported in 2050 is roughly 50\,TWh. The distributions present on the extreme left and right of the graphs correspond to the distributions of the output of interest for the samples of the GSA where the considered parameter takes a value in the bottom (left) or top (right) 15\% of its range. 

To improve the readability of this figure, I have updated the text zones to specify that these lines and distributions represent results of some specific samples of the GSA. Additionally, I have added a text zone to remind that the value between brackets is the Sobol' index of the considered parameter regarding this specific output of interest, and added this information in the caption of the figure:

\begin{figure}[htbp!]
\centering
\includegraphics[width=0.8\textwidth]{UQ_Methanol_samples_2.pdf}
\caption{Trend lines of the key parameters (and their Sobol' index) on the import of e-methanol in 2050. Around these lines, box plots point out the distribution of the output of interest for the extreme values (either bottom-15\% or top-15\%) of some parameters. The grey dashed line gives the value of the output of interest in the REF case. }
\label{fig:results_uq_samples_methanol}
\end{figure}

\section{Chapter 4 - Reinforcement Learning}
\label{Chap_RL}

\subsection{General}

\begin{mdframed}[style=comment] % Comment from the reviewer
{\color{orange} \textbf{Stefano}} - Overall, I suggest adding a diagram to Chapter 4, detailing what are the actions available to the decision-maker.
\end{mdframed}

\noindent Complementary to the text in the core of the manuscript, I have added a diagram to detail the agent's actions and how they impact the model: taken at the beginning of the time window to optimise (year $Y$), the four actions impact (i) the emissions of the system at the end of the time window (year $Y+10$) and, (ii-iv) the consumption of fossil gas, LFO and coal at years $Y+5$ and $Y+10$.

\begin{figure}[!htbp]
\centering
\includegraphics[width=0.8\textwidth]{Schematic_actions.pdf}
\caption{Actions available to the decision-maker. Taken at the beginning of the time window to optimise (year $Y$), the four actions impact (i) the emissions of the system at the end of the time window (year $Y+10$) and, (ii-iv) the consumption of fossil gas, LFO and coal at years $Y+5$ and $Y+10$. Unlike the first action that sets a target for the end of the time window, the last three aim at limiting the consumption of these fossil resources over the whole time window.}
\label{fig:Schematic_actions}
\end{figure}

\begin{mdframed}[style=comment] % Comment from the reviewer
{\color{orange} \textbf{Stefano}} \& {\color{purple} \textbf{Sylvain}} - How does this approach compare, for example, to stochastic programming?
\end{mdframed}

\noindent This work will be done for the subsequent paper on RL. The strategy would to implement a simplified case study (\ie smaller system, fewer uncertain parameters and more limited time horizon) where it would be ``computationally more affordable'' to test the stochastic programming approach and compare it with the optimal policy delivered by the RL approach. Consequently, regarding this comment, there has not been further modification brought to the manuscript.

\begin{mdframed}[style=comment] % Comment from the reviewer
{\color{orange} \textbf{Stefano}} \& {\color{violet} \textbf{Christophe}} - p.91, Table 4.2: how do you differentiate between actions that depend on the agent and exogenous uncertainties? More in general: how is uncertainty integrated in the RL framework?
\end{mdframed}

\noindent Both, agent's actions and exogenous uncertainties, change the value of some parameters of EnergyScope Pathway. However, where the sample of uncertainties is drawn once and for all at the beginning of the episode as explained in Section 1.2.1, the agent's actions are taken at the beginning of each 10-year time window. This is now clarified {\color{blue}at the beginning of Section 1.3.2 and the updated version of Figure 1.7, and its caption} (see Figure \ref{fig:Schematics_RL} in the present document):

\begin{mdframed}[style=manuscript] % Modification brought to the manuscript
Before starting an episode, a sample of uncertain parameters is drawn and affects the environment, EnergyScope Pathway, according to the methodology detailed in Section 1.2.1. 
\end{mdframed}

\begin{figure}[!htbp]
\centering
\includegraphics[width=0.8\textwidth]{Schematics_RL.pdf}
\caption{The Reinforcement Learning (RL) framework applied to the myopic optimisation of the energy transition pathway between 2020 and 2050. Here, the agent interacts with its environment, \ie the energy-system model on a limited decision window of 10 years. At the beginning of each episode, a different sample of uncertain parameters is drawn and affects the environment, EnergyScope Pathway, according to the methodology detailed in Section 1.2.1.}
\label{fig:Schematics_RL}
\end{figure}

\begin{mdframed}[style=comment] % Comment from the reviewer
{\color{orange} \textbf{Stefano}} - Explain more in details the different steps in the RL approach.
\end{mdframed}

\noindent Besides the general concepts/fundamentals detailed in Section 1.3.1, the application of the RL approach on the myopic optimisation of the energy transition of a whole-energy system is detailed in Section 1.3.2. Now, Figure 1.7 has been updated to clarify the difference between the uncertainty of parameters and the agent's actions as well as the interactions between the RL code (action, state, reward) and EnergyScope Pathway (optimisation of the concerned 10-year time window). I have also added a diagram to explain the impact that the agent's actions have on some parameters of the model. 

Given what was already in the manuscript and the aforementioned modifications, I have not added further information to detail the different steps of the RL approach.


\subsection{Graphs and results}

\begin{mdframed}[style=comment] % Comment from the reviewer
{\color{orange} \textbf{Stefano}} - Work on the graph of slide 19 (there is a misunderstanding on the fact that the grey curve is labelled as failure but they are failures in the future)
\end{mdframed}

\noindent To clarify this point, I have extended the caption of the related figures (see previous comments) and added two sentences in {\color{blue} Section 4.2.2}:

\begin{mdframed}[style=manuscript] % Comment from the reviewer
Since infeasible cases or those that overshoot the \ce{CO2} budget are discarded before reaching 2050 (see Section 4.1.2), the number of attempts that reach further steps in the transition progressively decreases. Consequently, the share of successful transitions compared to failures progressively increases with time. 
\end{mdframed}

\begin{mdframed}[style=comment] % Comment from the reviewer
\noindent {\color{orange} \textbf{Stefano}}, {\color{teal} \textbf{Stefan}} \& {\color{purple} \textbf{Sylvain}} - Graphs like the ones presented during the private defense (slides 22 and 24) should be better explained. In general, improve the presentation of the results, especially for the RL chapter.
\end{mdframed}

\noindent Beside extending the captions and clarifying the reasons why the number of total attempts decreases when going down the graph, as detailed in the previous comment, I have brought smaller modifications into the text to improve the presentation of the results, especially in the RL chapter.

\begin{mdframed}[style=comment] % Comment from the reviewer
{\color{teal} \textbf{Stefan}} - What is the potential of what we can learn from the results of the innovative methods like RL? We could go further in the analysis as we know we have to phase out coal, install renewables.
\end{mdframed}

\noindent First, I focus here below on each of the results sections of Chapter 4 highlighting the key conclusions already in the manuscript:

\begin{itemize}
\item \textbf{4.2.1 Reward and success}: (i) We have convergence of the learning process; (ii) Limiting success rate given the wide uncertainty ranges of some key parameters and the set of actions given to the agent; (iii) Near-term actions are needed.
\item \textbf{4.2.2 States}: (i) Validation of the formulation of the reward function (first respect the \ce{CO2} budget then minimise the cost); (ii) Reminder of the need for near-term actions; (iii) No-go zones in terms of share of renewables in the primary mix; (iv) Efficiency of the system does not give effective information for the agent to act upon.
\item \textbf{4.2.3 Actions}: (i) Limiting GWP and consumption of fossil gas are the keys; (ii) Even in a myopic perspective, limiting LFO is ineffective as it is phased out by the cost-driven model; (iii) Unlike coal that would still be part of the mix if the agent did not limit its consumption.
\item \textbf{4.3 Comparison with PF}: (i) Following \ce{CO2}-trajectory prescribed by PF would likely lead to failing the transitions with limited knowledge into the future; (ii) Succeeding myopic transitions relies more often on importing more electrofuels, mostly e-ammonia. 
\end{itemize}

Phasing out coal is, of course, necessary as it is a highly-polluting resource. However, what this work shows is that, whatever the stage in the transition, a policy limiting the use of coal will always be effective. However, to maximise the chance in succeeding the transition, the sooner the better. To insist on this aspect, this has been added {\color{blue}at the end of Section 4.2.3}:

\begin{mdframed}[style=manuscript] 
Whatever the stage in the transition, a policy limiting the use of coal will always be effective. However, to maximise the chance in succeeding the transition, the sooner the better.
\end{mdframed}

About the installation of renewables, not only it is (obviously) necessary to maximise the chance to succeed the transition (see Chapters 3 and 4) but it also provides robustness towards additional investments to make when deploying a roadmap through a (more realistic) myopic transition (see Chapter 5). This conclusion was already drawn by \citet{moret2020overcapacity} for the snapshot model and targeting the power sector only. Here, we extend this work and validate this conclusion for the whole transition pathway and addressing the additional investments in all the sectors.

Through the application of RL on the optimisation of the transition pathway, we are able to assess the effectiveness of a policy, which could be of interest to other researchers. This has been added {\color{blue}at the end of the conclusions of Chapter 4}:

\begin{mdframed}[style=manuscript] 
This framework allows future research to test other policies on other case studies and identify the key actions and the timing to take them through the transition.
\end{mdframed}

In conclusion, besides the results coming from the application to the specific case study of Belgium, a major added-value of this work is also to validate innovative approaches (RL and PCA) when applied to explore and optimise transition pathways under uncertainties. For instance, Section 4.2.2 shows that the reward function was properly defined to aim first at respecting the \ce{CO2} budget then minimising the cost. It also provides innovative ways to visually present results in the field of the optimisation of transition pathway under uncertainties that suffers from the (overwhelming) curse of dimensionality. Via distributions or principal components, this work summarises a lof of information into graspable visuals that allow drawing effective conclusions such as the identification of sweet-spots and no-go zones.

Given these explanations and what is already written in the manuscript, I have not brought other modification to the text.

\begin{mdframed}[style=comment] % Comment from the reviewer
{\color{orange} \textbf{Stefano}} - p. 89: Figure 4.3 seems to deliver an important message (early action is needed to ensure we meet climate goals), but it is not clearly explained. Improve captions / add axes. 	Where do I see the ``tipping points'' in this figure?
\end{mdframed}

\noindent Like other figures in this Chapter, Figure 4.3 represents distributions of features spreading over the x-axis. In this case, these are distributions of the rewards collected by the agent at the end of each episode of its learning process. For this reason, labelling the axis does not seem obligatory as the x-axis is nothing else than the feature (here, the reward) and the y-axis has no physical meaning expect representing how often the feature of interest took a certain value/range of values. Compared to the other similar figures of the chapter, I have not labelled the x-axis as, for Figure 4.3, the reward is the only feature represented whereas two or more features are gathered on the same figure.

With this in mind and what is already written in the caption of the figure and the core of the text, we can infer from the right hand side that 2040 is the ``tipping year'' as it represents the year at which the failures occur more often, compared to the other representative years. 

Given these explanations and what is already in the text, I have not brought modifications to the manuscript in this regard.

\begin{mdframed}[style=comment] % Comment from the reviewer
{\color{orange} \textbf{Stefano}} - p. 90: What does it mean to succeed/fail in 2030? And why are the number of attempts reduced going down in the graph?
\end{mdframed}

\noindent As discussed during the private defense, the grey curves represent the distribution of occurrences failing (\ie infeasible optimisation or overshooting \ce{CO2} budget) by 2050 at the latest. The number of remaining attempts reduce as prematurely ended episode are removed from the further counts. The purpose of displaying these numbers of ``Total attempts'' is to explain why the areas below the orange curves increase relatively to the grey curves. These two aspects are now explicitly explained {\color{blue}in the caption of Figure 4.4 and the other similar figures}:

\begin{mdframed}[style=manuscript] % Modification brought to the manuscript
Exploration of the state space over the learning process: distribution of occurrence of cumulative emissions (left) and costs (right). \textbf{Number of remaining attempts decrease with time since infeasible problems and solutions overshooting the \ce{CO2} budget are discarded prematurely, \ie before reaching 2050. Besides infeasible problems, distributions labelled as ``Failure'' represent the attempts that overshot the \ce{CO2} budget by 2050 at the latest.} The majority of successful transitions have cumulative emissions much lower than the \ce{CO2} budget and are cheaper than the REF case. 
\end{mdframed}

\begin{mdframed}[style=comment] % Comment from the reviewer
{\color{orange} \textbf{Stefano}} - p. 94: Figure 4.6 seems to be less revealing than other figures. Also the figure caption seems to contradict the first statement following the figure. It would be important to clarify: which are the actions that emerge to actually matter?
\end{mdframed}

\noindent Indeed, Figure 4.6 is less revealing as it shows no big trend in terms of neither impacting actions nor their timing in the transition.  It is detailed in the first paragraph of Section 4.2.3 after which the figure should appear in the manuscript. I have worked to make Figures as close as possible to their related paragraph. However, in some cases, several other paragraphs come between them. To identify the actions that emerge to actually matter, we need to check if they are binding or not, as detailed in the subsequent paragraphs and in Figure 4.7.

Given these explanations and what is already present in the text, I have not brought further modification to the manuscript in this regard, except minor adjustment to get Figure 4.6 as close as possible to its related paragraph.


\subsection{Others}

\begin{mdframed}[style=comment] % Comment from the reviewer
{\color{teal} \textbf{Stefan}} \& {\color{purple} \textbf{Sylvain}} - p. 97 - The agent can reach systems that are cheaper than any other solution obtained by the perfect foresight approach. How can this be possible?
\end{mdframed}

\noindent This was observed on the annual system cost which is not the objective function of the Pathway model. It is more relevant to look at the total transition cost which is, logically, lower for the PF approach compared to myopic when facing the same sample of uncertain parameters. To clarify this aspect, I have removed the paragraph (and the graph) about the comparison of the annual system cost and added instead a paragraph (and a graph) about the comparison of the total transition cost {\color{blue}in Section 4.3}:

\begin{mdframed}[style=manuscript] % Modification brought to the manuscript 
Looking at the total transition cost, the combination of the agent's actions and favourable economic conditions (see Section 4.2.2) make the myopic transitions cheaper, on average, than the PF cases (see right side of Figure \ref{fig:Gwp_pathway_total_tran_cost}). This is also due to the fact that the perfect foresight approach always finds a solution even in worst conditions such as high cost of purchasing resources and high EUD. This explains the wider variability of the results too. However, with the same sample of uncertain parameters, given the assumed full knowledge over the whole time horizon, PF naturally results in a cheaper transition than its myopic equivalent.
\end{mdframed}

\begin{figure}[!htbp]
\centering
\includegraphics[height=4cm]{Gwp_pathway_core.pdf}
\includegraphics[height=4cm]{Transition_cost_comp_2.pdf}
\caption{Comparison of \ce{CO2}-emissions pathways (left) and total transition cost (right) from the perfect foresight optimisation under uncertainties and the RL-based myopic optimisation. Myopic transitions succeed with a more drastic reduction of emissions in the short-term and, on average, more favourable economic conditions.}
\label{fig:Gwp_pathway_total_tran_cost}
\end{figure}

\begin{mdframed}[style=comment] % Comment from the reviewer
{\color{violet} \textbf{Christophe}} - The concept of binding constraint was not clear.
\end{mdframed}

\noindent
Besides the information given in Section 4.2.3 of the manuscript, I do not see further explanations that could clarify more the concept of binding constraint in a Linear Programming (LP) problem. Consequently, regarding this comment, there has not been further modification brought to the manuscript.

\section{Chapter 5 - Principal Component Analysis}
\label{PCA}

\begin{mdframed}[style=comment] % Comment from the reviewer
{\color{orange} \textbf{Stefano}} - p. 101 - you mention different definitions of ``robustness'': which one do you use in your work?
\end{mdframed}

\noindent This work assesses the robustness of the decision-making process in two different ways. On the one hand, using a Reinforcement Learning (RL) approach, it allows highlighting the key actions to take to support a myopic transition and maximise the chances to respect the emissions-budget. On the other hand, after drawing a robustness-metric based on a Principal Component Analysis (PCA) approach, this work investigates how much a planned roadmap is sensitive to additional investments required in a myopic transition with progressively unveiled uncertainties. In other words, we use two different definitions of robustness: (i) the ability of a policy to maximise the chances to succeed a myopic transition under uncertainties to meet a \ce{CO2} budget, (ii) the ability of a technological roadmap to limit the investment into additional capacities when tested in myopic conditions. 

These two definitions are now clearly introduced {\color{blue}in the introduction of the manuscript}:

\begin{mdframed}[style=manuscript] % Modification brought to the manuscript
Chapter 4 focuses on the robustness of a policy as its ability to maximise the chances to succeed a myopic transition under uncertainties to meet a \ce{CO2} budget.

Chapter 5 assesses the robustness of different technological roadmaps as their ability to limit the investment into additional capacities when tested in myopic conditions. 
\end{mdframed}

{\color{blue}at the end of the introductory paragraph of Chapter 1}:

\begin{mdframed}[style=manuscript] % Modification brought to the manuscript
Finally, an agent-based Reinforcement Learning (RL) approach is detailed to address the robustness of a policy in uncertain transitions with limited vision in the future. This chapter ends with a PCA-based approach to draw a robustness metric and assess the ability of technological roadmaps to limit the investment into additional capacities.
\end{mdframed}

{\color{blue}in the second to last paragraph of the contributions of Chapter 1}:

\begin{mdframed}[style=manuscript] % Modification brought to the manuscript
The third principal methodological contribution is the use of Principal Component Analysis (PCA) to assess the robustness of a technological roadmap. Given the uncertainties and the timespan of the transition, this approach allows highlighting the main ``directions'' of variation of the system design (\ie the installed capacities). After this step of identification, strategies can be projected on these directions to see how robust they are to the need of additional capacities.
\end{mdframed}

In the second paragraph of Chapter 5, we list different definitions of robustness. In this work, we consider indeed ``static'' robustness as we aim at either getting a policy that maximises the chance to succeed the transition or a technological roadmap that limits the need to invest into additional capacities. It is now clarified {\color{blue}in the second paragraph of Chapter 5}:

\begin{mdframed}[style=manuscript] % Modification brought to the manuscript
Where Chapter 4 addressed the static robustness of a policy as its ability to maximise the chances to succeed a myopic transition, the objective of this chapter is to apply the method described in Section 1.4 to deal with the static robustness of pathway technological roadmaps, defined as their ability to limit the need to invest into additional capacities.
\end{mdframed}

Finally, I have gone through the manuscript and made sure that RL-based approach was addressing the robustness of \textbf{policies} (\ie sequences of actions) whereas the PCE-based approach rather focuses on the robustness of \textbf{technological roadmaps}. 

\begin{mdframed}[style=comment] % Comment from the reviewer
{\color{violet} \textbf{Christophe}} - When using the word ``component'', there seems to be a confusion and it is not always easy to understand if you refer to the vector or the coefficient related to one of the original variable.
\end{mdframed}

\noindent The confusion probably comes from the fact a Principal Component (PC) actually represents an eigenvector of the covariance matrix. This vector is, by definition, composed of \textbf{components}, each of them being a coefficient related to a specific original variable. Parts where the word ``component'' is not directly linked to ``vector'', ``eigenvector'' or ``PC'' are those that could mislead the reader. Modifications have been brought to these parts.

{\color{blue} End of second paragraph of section 1.4.1}:

\begin{mdframed}[style=manuscript] % Modification brought to the manuscript
Moreover, this means that \textbf{the coefficient} $\alpha_{ki}$, \ie the component of $\bm{\alpha}_{\mathbf{k}}$ related to the $i^{\text{th}}$ original variable, $x_i$,  gives its weight in the $k^{\text{th}}$ PC, \ie $z_k$. 
\end{mdframed}

\section{Conclusion}
\label{Conclusion}

\begin{mdframed}[style=comment] % Comment from the reviewer
{\color{teal} \textbf{Stefan}} - The conclusion is a summary and should rather be a discussion.
\end{mdframed}

\noindent I decided to split the conclusion into four blocks of paragraphs. The first two (from page 117 to the top of page 120, in the version of the manuscript submitted before the private defense) are indeed a summary of (1) what we have done in this thesis (case study and methodology) and (2) the results we have got. The two last blocks, however, take a step back from the results and aim at drawing main conclusions, discussing them and providing decision-makers with further insightful information and methods to plan the energy transition. Consequently, I have not brought modification to the manuscript in this regard.

\clearpage
\def\bibfont{\scriptsize}
\bibliography{../bib_thesis.bib}
\normalsize

\end{document}