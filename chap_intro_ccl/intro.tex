It has been proven that the climate change (among other environmental challenges) is (mostly) due to the concentration of anthropogenic \gls{GHG} in the environment \cite{IPCC_CO2_budget}. This concentration resulting from the cumulative emissions (and subtractions), usually expressed in kt$_{\ce{CO2},\text{eq}}$, over time could be developed as an adapted, \ie less economy-oriented, version of the original Kaya identity \cite{kaya1997environment}:

\begin{equation}
\label{eq:equality_GHG}
\mathrm{GHG} =  \frac{\mathrm{GHG}}{\text{Primary energy}} \times \frac{\text{Primary energy}}{\mathrm{EUD}}\times \frac{\mathrm{EUD}}{\text{Population}} \times \text{Population}\\
 \end{equation}

\noindent
where the first term represents the \gls{GWP} of the primary energy mix, the second is the inverse of the efficiency and the third could stand as the energy intensity per capita. Such an identity, mathematically-correct though, is criticized for the arbitrary choice of variables, the non-independence of them usually leading to the rebound effect and its global/encompassing approach that does not translate properly the heterogeneity of the situation \cite{IPCC2000}. However, Eq. \ref{eq:equality_GHG} has the merit to highlight three levers of action that should be activated to reduce the \gls{GHG} emissions and, consequently, favour the transition. Besides the question of the total population and its growth \cite{dodson2020population,scovronick2017impact}, these three levers of actions are: renewables, efficiency and sufficiency aiming at reducing the first, the second and the third terms on the right-hand side of Eq. \ref{eq:equality_GHG}, respectively. The latter, explicitly mentioned by the IPCC for the first time in 2022 \cite{IPCC2022}, is defined by \citet{lage2023citizens} as ``a strategy for reducing, in absolute terms, the consumption and production of end-use products and services through changes in social practices in order to comply with environmental sustainability while ensuring an adequate social foundation for all people''. Altough this finds a growing interest in the scientific community \cite{o2018good}, it requires, maybe more than the two other levers, interdisciplinarity \cite{schmidt2015interdisciplinary}, \ie the combination of multiple academic disciplines like sociology, psychology or politics, that are out of the scope of my expertise, and, consequently, this thesis. However, the work developed in the present manuscript aims at providing support to such interdisciplinary projects to assess sufficiency policies. More within the grasp of the engineering world, this thesis rather focuses on the first two terms of Eq. \ref{eq:equality_GHG}, \ie renewables and efficiency. This aligns with the current European policies binding the Member States of the European Union. For instance, the Renewable Energy Directive (RED) III, published in October 2023 \cite{REDIII}, highlights that ``the Union’s climate neutrality objective (by 2050) requires a just energy transition which leaves no territory or citizen behind, an \textbf{increase in energy efficiency} and significantly \textbf{higher shares of energy from renewable sources} in an integrated energy system'' (\ie 42.5\% of the Union's gross final consumption of energy by 2030). 





It aims at providing decision-makers with new methods and informed policies accounting for the intrinsic uncertainties of the future. 



%!TEX root = ../thesis_main.tex
%!TEX encoding = UTF-8 Unicode

In their 2030 Agenda, United Nations have worked on identifying 17 \gls{SDGs} as a plan of action for society (or people), environment (or planet) and economy (or prosperity) \cite{un_sdgs}. 
%
%As Fatih
%
%On top of the depletion of (economically and environmentally) easy to harvest conventional fossil resources, the most urgent motivation for energy transition is the growing threat of climate change.  This change jeopardizes most of the \gls{SDGs} listed by the United Nations . Besides their environmental pillars (\ie 13-climate action, 14-life below water and 15-life on land)\cite{vinuesa2020role}, succeeding such a transition would contribute in reaching others of these goals like affordable 

General introduction to the need of energy transition\\

How electrofuels will be part of the solution. Here include part of the terminology paper \url{https://www.frontiersin.org/articles/10.3389/fenrg.2021.660073/full} to clearly set the context and make sure the readers understand what's behind electrofuels and emphasize at the end that in our case, we consider electrofuels as the ones from renewable sources (with a 0-gwp).\\

From this need, develop a thread like I did in my FRIA application, we need whole-energy system model to give more insights to policymakers\\

Listing some examples of situations where uncertainties have not been considered and ended up in over-cost/waste of time/waste of..., highlight the need as well to consider uncertainties when we want to advise policymakers.\\

As the question of policymakers is not only what to do but how to do it, we need to address the optimisation of policies. This would introduce the RL part


\textbf{In a more sustainable future, some of the energy carriers, currently produced mostly from fossil resources, will still consist of hydrocarbons (\eg e-methane or e-methanol). This is why this paper rather uses ``defossilisation" rather than ``decarbonisation" as carbon will still play a key role in a carbon-neutral energy transition \cite{mertens2020carbon}.}



