%!TEX root = ../thesis_main.tex
%!TEX encoding = UTF-8 Unicode

Our society In their 2030 Agenda, United Nations have worked on identifying 17 \gls{SDGs} as a plan of action for society (or people), environment (or planet) and economy (or prosperity) \cite{un_sdgs}. 
%
%As Fatih
%
%On top of the depletion of (economically and environmentally) easy to harvest conventional fossil resources, the most urgent motivation for energy transition is the growing threat of climate change.  This change jeopardizes most of the \gls{SDGs} listed by the United Nations . Besides their environmental pillars (\ie 13-climate action, 14-life below water and 15-life on land)\cite{vinuesa2020role}, succeeding such a transition would contribute in reaching others of these goals like affordable 

General introduction to the need of energy transition\\

How electrofuels will be part of the solution. Here include part of the terminology paper \url{https://www.frontiersin.org/articles/10.3389/fenrg.2021.660073/full} to clearly set the context and make sure the readers understand what's behind electrofuels and emphasize at the end that in our case, we consider electrofuels as the ones from renewable sources (with a 0-gwp).\\

From this need, develop a thread like I did in my FRIA application, we need whole-energy system model to give more insights to policymakers\\

Listing some examples of situations where uncertainties have not been considered and ended up in over-cost/waste of time/waste of..., highlight the need as well to consider uncertainties when we want to advise policymakers.\\

As the question of policymakers is not only what to do but how to do it, we need to address the optimisation of policies. This would introduce the RL part





