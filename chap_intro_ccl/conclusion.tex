%!TEX root = ../thesis_main.tex
%!TEX encoding = UTF-8 Unicode
I took here the same sections as in Gauthier's thesis
\section*{Thesis contributions}
\addcontentsline{toc}{section}{Thesis contributions}
Insist here on the methodological added value of the thesis

\section*{Limitations}
\addcontentsline{toc}{section}{Limitations}
\begin{itemize}
\item Due to the formulation of the salvage value, some technologies remain in place whereas they are not used anymore (see Appendix B.3, e.g. coal boilers and naphtha-cracker). It's better for the system to keep unused assets to recover part of their salvage value than prematurely decommissioning them.
\end{itemize}

\section*{Application outcomes}
\addcontentsline{toc}{section}{Application outcomes}
What this new methodology has brought over when applied to the case of Belgium

\section*{Recommendations and guidance}
\addcontentsline{toc}{section}{Recommendations and guidance}
What to do then for policymakers, how to use the tool

\section*{Perspectives}
\addcontentsline{toc}{section}{Perspectives}
List the future works to build upon the thesis

\begin{itemize}
\item \textbf{Word about sufficiency} oui mais c'est sans doute celle qui est le plus enviable car la moins risqué à tenter d'atteindre, les solutions mirages technologiques, si on y croit trop on met tout la dessus et si ca foire, on est encore plus dans la merde pcq la direction est mauvaise, la solution sobriété, meme si jamais atteint les efforts pr l'atteindre ne seront pas contre productif
\item \textbf{Word about availability of electrfuels} Voir mémoire Ced et Simon
\item Extract a roadmap representative of a multiple-run UQ or RL process.
\end{itemize}


