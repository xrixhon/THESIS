%!TEX root = ../thesis_main.tex
%!TEX encoding = UTF-8 Unicode


% WHAT WE DID (HOW?)
The general objective of this thesis has been to investigate the impact of uncertainties and limited foresight on the transition pathway of a whole-energy system, accounting for multi-energy carriers and multi-sectors, subject to \ce{CO2}-budget without prescribed trajectory for the reduction of the emissions.  We have pursued this objective on the case study of Belgium with a focus on the role that renewable electrofuels such as e-ammonia, e-methane, e-methanol and e-hydrogen, could play in this densely-populated and highly industrialised country with limited local renewable potentials. To have 66\% chance of limiting warming to 1.5°C \cite{IPCC_CO2_budget} and using the ``grandfathering'' sharing principle of the global budget \cite{knight2013grandfathering}, the \ce{CO2}-budget of the 2020-2050 transition of Belgium has been set to 1.2\,Gt$_{\ce{CO2},\text{eq}}$. This budget represents 10 years of the current level of emissions.

To optimise such a system, we have developed EnergyScope Pathway that minimises the total transition cost of the system from 2020 to 2050 and implemented the myopic approach. This approach considers a sequence of 10-year time windows with an overlap of 5 years rather than one global optimisation over the 30 years of the transition. This allows assessing the realism of the decision making process where knowledge about the future is progressively unfold rather than assumed to be fully known from the beginning of the transition as in the perfect foresight (PF) approach. 

The future of a national whole-energy system is subject to a plethora of uncertainties such as the cost of purchasing resources, their availabilities, the \acrfull{CAPEX} of technologies or the level of demands. In this thesis, we have used the uncertainty characterisation approach developed by \citet{moret2016strategic} to provide the range of variation of the uncertain parameters. To propagate and quantify the impact of these uncertainties on the outputs of the model, we have used the \acrfull{PCE} method has been used as implemented in the Coppitters' RHEIA framework \cite{coppittersthesis}.

The \acrfull{RL} approach was adopted to explore the myopic pathways respecting a \ce{CO2}-budget equivalent to 10 years of current emissions, without \ce{CO2}-trajectory. In this framework, an agent takes actions at the beginning of each time window to limit the emissions of the system, the \textit{environment}, and the consumption of fossil fuels. Starting from the initial state of the energy system in 2020, the agent takes every five years a set of actions until reaching 2050. Repeating the whole transition with different sequences of actions-states allow the agent coming up with an effective policy towards sustainability, considering the variation of the parameters of its environment.

Finally, we have assessed the robustness of technological roadmaps tested in myopic transitions. To do so, the approach of \acrfull{PCA} was mobilised to identify the technologies that are more sensitive to uncertainties and that are more likely to impact the overall variance of the transition design. Projecting the results of myopic transitions on these principal components of the transition pointed out which roadmap is more robust and less likely to require additional investments along the transition.\\

% THE RESULTS WE GOT (WHAT?)
Given the \ce{CO2}-budget to stick to a +1.5$^{\circ}$C increase of global temperature and the presumably high levels of demands,  the cost-optimum transition of Belgium should be to first opt for more efficiency to reduce by one third the primary energy consumption electrofuels. Then comes the full deployment of local solar and wind energies that would represent about 30\% of the mix by 2050. To fill the gap left by limited potential of local renewables, the electrofuels are found to be the third pillar of the Belgian transition. 47\% of the variation of the total transition cost are due to the uncertainty on the cost of purchasing these energy carriers, before the industrial end-use demands (23\%), the interest rate (12\%) and the cost of purchasing fossil fuels (6\%). 

E-methanol is found to be the key-molecule to import in the mid-term to defossilise the non-energy demand and substitute naphtha-crackers to produce on average 43\,TWh \acrfull{HVC}. The uncertainty on this demand is the most impacting parameter (80\%) on the variation of the quantity of imported e-methanol. 

In the longer-term, importing e-ammonia becomes crucial to supply its own non-energy demand ($\sim$ 10\,TWh) and, before all, \acrfull{CCGT} to provide flexibility in the production of electricity in competition with nuclear \acrfull{SMR}. For this reason, where the variation of the quantity of imported e-ammonia is mostly driven by the uncertainty on its cost of purchasing (50\%), the uncertainty on the availability of nuclear \gls{SMR} by 2040 is the second most impacting parameter (20\%). 

To a smaller extent, e-methane help defossilising the industrial boilers and \acrfull{CHP}. E-methane imports is mostly impacted by industrial demands (45\%). 

By 2050, e-hydrogen is only imported from abroad in the wide majority of the transition pathways under uncertainties generated during this work.  This molecule supports the defossilisation of the freight-transport by trucks. The variation of the imported quantity of e-hydrogen is mainly caused by the uncertainty on the CAPEX of fuel cell vehicles (25\%) and its cost of purchasing (20\%). In 12\% of the transition pathways, up to 10.5\,GW of electrolysers are installed when local renewable electricity is abundant, \gls{SMR} is available and imported e-hydrogen is expensive. These electrolysers would locally produce up to 27\,TWh of e-hydrogen. 

Through the Reinforcement Learning-based exploration of myopic transitions, we pointed out that near-term actions were mandatory to hope succeeding this ambitious transition.  Beyond 2030, if the right actions have not been taken, the cumulative emissions are likely to overshoot the \ce{CO2}-budget. Besides limiting the use of coal at any cost, limiting the consumption of fossil gas was needed at the early stages. Once the fuel switch has been operated from fossil gas to e-methane, limiting the overall emissions is the most effective action to stick to the direction towards sustainability and encourage the import of renewable energy carriers. Regarding the information on which the \gls{RL}-agent determines its actions, the system overall efficiency provides less informative insight than the share of renewable energy carriers in the primary energy mix. The latter shows intermediate milestones to reach, 62\%-share by 2030, to have higher chance to succeed the transition by 2050. Below this threshold, these chances of success are much more limited, the so-called no-go zones. Finally, due to limited foresight into the future and its uncertainties, successful myopic transitions rely more on importing renewable electrofuels than the perfect foresight approach. This is even more the case for e-ammonia that is in general twice more imported by 2050 in myopic than PF transitions.

In the case of Belgium, the uncertainty on the maximum installed capacity of solar \gls{PV}, which is supported by the electrification of the industrial and decentralised heating sectors, drives 57\% of the variance of the technological mix through the transition. This affects the entire transition. On the contrary, other key contributions to this variance are more focused on tipping years. For instance, the low-temperature heating sectors shift from oil and gas decentralised boilers to electric \acrfull{HP} and \acrfull{DHN} from 2025 to 2035. In the non-energy sector, the shift between naphtha-crackers and \acrfull{MTO} takes place in 2025.

Finally, testing roadmaps resulting from perfect foresight optimisations in myopic conditions has shown that cases relying or not on 6\,GW of \gls{SMR} by 2040 onward provide the same level of robustness in terms of additional capacities to install. However, despite representing an additional 2.6\% cumulative CAPEX over the transition, investing as soon as possible in local \gls{PV}, the electrification of heating sectors and more efficient technologies such as \gls{CHP} provides significantly better level of robustness (between 33\% and 49\%).\\


% WHAT TO DO WITH THAT (SO WHAT?)
In the context of the energy transition, this thesis aimed at providing decision-makers with information and new methods considering the intrinsic uncertainties of the future. Even though the following messages result from studies on Belgium, the trends can be transposed to other countries with a high demand and low renewable potentials such as the Netherlands or Germany. 

Besides a wider electrification of the system and the integration of more efficient technologies, a robust policy consists in investing in electrofuels. Chemically similar to their fossil-based equivalents, they can benefit from the current infrastructures and provide flexibility to the mismatch between the production of renewable electricity and its consumption. They would be imported from countries ``in excess'' of renewable electricity from wind or solar such as Australia \cite{Australia_efuels}, Chile \cite{Chile_efuels} or Spain \cite{Spain_efuels} to provide seasonal storage of electricity. Already stated in the analysis of the snapshot model of \citet{rixhon2021role}, this work confirms through the optimisation of the whole transition pathway that we should aim at reducing the uncertainty on the cost of purchasing these energy carriers. On the road to a robust whole-energy system in Belgium, the policy makers, the industries, and academia should invest in projects to produce and use these fuels or developing the exchange networks with neighbouring countries. This would provide significant reduction in the variation of the total transition cost and robustness towards uncertainties progressively unfolding with the future. 

About the use of the electrofuels, the two molecules imported in largest quantity, e-methanol and e-ammonia, would aim at defossilising the non-energy and providing flexibility in the production of electricity, respectively. The main use of e-hydrogen is in the freight transport via trucks and e-methane would supply the industrial sectors through \gls{CHP} and boilers.

Also brought up as another ``unicorn'' solution to the problem of the energy transition, nuclear \gls{SMR} would be a cost-competitive option against the import of electrofuels, especially e-ammonia and e-methane, but would still support the deployment of local \acrfull{VRES} such as solar and wind. This confirmed that investing as soon as possible in local renewables is one pillar of the energy transition, especially when accounting for uncertainties and with the goal to respect an ambitious \ce{CO2}-budget of the transition. 

On a methodological point of view, this thesis developed novel approaches to explore myopic pathway transitions, highlight intermediate milestones and no-go zones towards sustainability and assess the robustness of transition technological roadmaps. Keeping this work open-source allows the scientific community and the decision-makers to use these methods and adapt them on their specific case study. \\


As the research in the field of the energy transition is a never-ending story, we conclude with perspectives that could benefit from the methods and the results developed in this thesis.  Future studies could focus on different scenario analyses, the development of EnergyScope Pathway or the approaches applied to it such as \gls{RL}. 

The Belgian transition pathways explored in this thesis massively relied on the early deployment of local \gls{VRES} and import of electrofuels. In some of these pathways, in a 5 to 10 year time window, the current capacities of \gls{PV} and wind turbines are multiplied by 10 and 6, respectively. This is much faster than the multiplication by 2 and 2.5 of PV and wind capacities, respectively, from 2024 and 2034 projected by Elia, Belgium Transmission System Operator (TSO),  in their adequacy analysis \cite{Elia_2024_2034}. On their side, the amount of imported electrofuels could be 10 to 20 times more than the potential that Belgium could benefit from exporting countries, given the already signed agreements \cite{lefebvre2022electrofuel}. As the objective of this work was to explore as much as possible transition pathways, we have decided not to put further constraints on these energy carriers. One future work would be to constrain the potential of local renewables to a more progressive deployment or the availability of the electrofuels. This could provide different pathways where electrofuels would be produced more locally using carbon-capture technologies. This could also show the limit of the Belgian energy to meet such an ambitious \ce{CO2}-budget in its energy transition. 

To assess more accurately the boundary conditions of the Belgium in its transition pathway, a future work could be to merge the EnergyScope Pathway with EnergyScope Multi-Cells \cite{thiran2023validation}. Rather than model a single country and the rest of the world as the exterior, the Multi-Cells optimises multiple interconnected regions. At a European scale, Belgium could benefit from energy exchanges with countries where the renewable potential is more important such as France or Spain. 

Rather than limiting the availability of renewable resources, one could question the level of demands and address the question of the sufficiency. For a growing part of the scientific community \cite{o2018good}, sufficiency is seen as a no-regret lever of action to reduce the anthropogenic \gls{GHG}. A study could consist in analysing the level of demands, within their respective sector, we should reach to meet the \ce{CO2}-budget and limit the installation of solar PV, wind turbines or nuclear energy or the import of electrofuels.

On the methodological side, several perspectives could be considered about the whole-energy system model or the implementation of \gls{RL}.

The cumulative \acrfull{GHG} emissions of the transitionhas been constrained based on the use of the resources. However, in a system reducing drastically its emissions, indirect emissions such as the ones related to the construction and dismantling of assets, would have a much higher share of the total emissions. Depending on the level of emission reduction, these indirect emissions could be one to three times higher than the direct emissions \cite{blanco2020life}. To accurately integrate these emissions in the model and account for them in the \ce{CO2}-budget, \gls{LCA} can compute these indirect emissions. On top of this, \gls{LCA} could provide data to assess other impact indicators besides climate change \cite{astudillo2018integrating}. Based on this multi-criteria approach, another thread of future work would be to implement multi-objective optimisation \cite{dubois2023multi}. In this method, objectives other than the sole total transition cost such as the ones identified as planetary boundaries \cite{richardson2023earth} would be minimised.

Regarding the \gls{RL} approach, two threads of future development have been identified: reward shaping and multi-fidelity.  First, the reward function could be adapted to incentivise more the successful transitions. Then, rather than giving a sparse reward at the end of the episode, reward shaping provides additional, intermediate rewards during the learning process based on various heuristics, domain knowledge, or problem-specific insights \cite{hu2020learning}. Reward shaping aims at accelerating the learning or guiding the agent towards achieving the desired behaviour more efficiently. In our case, the agent receives a sparse reward signal, indicating success or failure at the end of an episode. However, this sparse reward signal may not provide enough information for the agent to learn effectively, leading to slow convergence or difficulty in learning the optimal policy. Reward shaping addresses this issue. These intermediate rewards can help guide the agent towards desirable states or actions, making the learning process more efficient and effective. However, reward shaping should be applied with caution, as poorly designed reward functions can lead to unintended consequences such as suboptimal policies, reward hacking, or overfitting to the shaped rewards rather than learning the underlying task.  

Given the similarities between the policy learned on the monthly (coarse) and the hourly (fine) model, another way to improve the efficiency of the \gls{RL} process would be to implement transfer knowledge \cite{mann2013directed}, or even multi-fidelity \cite{cutler2014reinforcement}.  The first step, transfer learning, would consist in starting the learning process on the monthly and faster model, refine the action space, and continue learning on the more accurate hourly model.  More than this unidirectional transfer of information, \citet{cutler2014reinforcement} developed the Multi-Fidelity Reinforcement Learning (MFRL) framework that defines rules for when the agent should move up to a higher fidelity environment, as well as moving down in fidelity before over-exploring in a more expensive optimisation.  This would result in an agent that can, on the one hand, use information from the monthly model to limit the exploration in the hourly model, and, on the other hand, exploit the data of the hourly model to update the policy learned on the monthly one.\\

Nowadays, the development on solar and wind energies supported by other technical solutions such as electrofuels exchange market or nuclear energy, is not fast enough to stick to the +1.5$^{\circ}$ C of global warming. Even though it represents a growing part of the literature, sufficiency is rarely brought up by decision-makers as a solution. At the same time, whole-energy system models and related methodologies do not claim to predict the future but rather aims at quantifying options under several assumptions. In this sense, this work provided insight about possible transition pathways for Belgium to bring the technical dimension into discussions and decisions that should account for more interdisciplinarity.