%!TEX root = ../thesis_main.tex
%!TEX encoding = UTF-8 Unicode

C’est une bonne situation, ça, doctorant? Moi je ne crois pas qu’il y ait de bonne ou de mauvaise situation. Moi, si je devais résumer ma thèse aujourd’hui avec vous, je dirais que c’est d’abord des rencontres, des gens qui m’ont tendu la main, peut-être à un moment où je ne pouvais pas, où j’étais seul chez moi. Et c’est assez curieux de se dire que les hasards, les rencontres forgent un docteur… Parce que quand on a le goût de la chose, quand on a le goût de la chose bien faite, le beau geste, parfois on ne trouve pas l’interlocuteur en face, je dirais, le miroir qui vous aide à avancer. Alors ce n’est pas mon cas, comme je le disais là, puisque moi au contraire, j’ai pu ; et je dis merci aux personnes qui m'ont aidé dans cette aventure, je leur dis merci, je chante ces personnes, je danse ces personnes… Je ne suis que reconnaissance ! 

% Promoteurs
Bien qu'un seul nom figure en couverture de cet ouvrage, les pages qui suivent seraient tristement restées vierges sans tout un environnement humain propice à la réflexion scientifique et à mon épanouissement personnel. Merci avant tout à mes deux promoteurs de thèse, mes deux ``pères académiques'', Hervé Jeanmart et Francesco Contino. Merci de m'avoir fait confiance dès le premier jour, donner la liberté de mener à bien mes recherches en toute autonomie tout en veillant à me prêter vos oreilles attentives et me prodiguer vos conseils avisés lors de nos échanges. Merci pour le temps que vous avez consacré lors de nos réunions pour résoudre les casse-têtes de ma thèse ainsi que ceux posés sur la table de Francesco. Merci à Hervé pour son pragmatisme et sa spontanéité et à Francesco pour son empathie et son goût du détail pour toujours chercher le dernier carat, à la limite parfois de la diptérophilie.

% Membres de jury
Merci aux membres de mon jury et encore plus aux membres de mon comité d'accompagnement, Sylvain Quoilin et Stefano Moret, pour le temps qu'ils ont consacré à la lecture de mon manuscrit et surtout pour les remarques toujours bienveillantes et constructives adressées lors de nos réunions et de la défense privée. Celles-ci m'ont permis d'apporter des améliorations majeures au texte et de mettre en exergue de futures pistes de recherche. 

% UCLouvain + BEST + Véro
Pour avoir rendu lucrative cette aventure humaine et intellectuelle, je tiens à remercier l'UCLouvain mais aussi le SPF Economie au travers du Fonds de transition énergétique et du projet BEST. J'en profite pour dire un énorme merci à la coordinatrice de ce projet, Véronique Dias. Outre mes deux pères académiques, tu auras été une vraie mère, toujours aux petits soins pour moi et les autres chercheurs et chercheuses. Merci pour ton temps et ton investissement afin de faire de ce projet une franche réussite.

% Collègues en commençant par Gauthier et Diederik
Merci ensuite à mes parrains, Gauthier et Diederik. Vous m'avez catapulté dans cette aventure et m'avez toujours accompagné pour explorer de nouveaux horizons. Merci à Marion et Ignace pour votre écolage dans l'univers du Reinforcement Learning. Enfin, outre les tontons (parfois) un peu (trop) grincheux, je ne saurais qu'assez remercier tous mes frères et soeurs/cousins et cousines de cette belle et grande famille du TFL. Dans le désordre, merci Alice pour ton humour et tes goûts musicaux ``discutables'' et avoir tenté tant bien que mal de m'initier à la science de la musique et à la culture flamande. Merci Arnaud pour ton amitié de longue date, ta disponibilité, tes relectures et tes opinions toujours magistralement construites. Merci coach PB de m'en avoir fait baver et de m'avoir poussé à dépasser mes limites lors de nos nombreuses courses. Merci Youyou d'y avoir supporté nos discussions de darons (et de me botter désormais les fesses). Merci Mamasita Aurel pour ta good vibe et ton eau pet', Martinos pour la co-création du label, Paolito pour ta force tranquille, Antoine aka Jabiru pour ton abonnement au MR magazine, p'tit Louis pour avoir toujours si bien veillé à ma roue arrière, Zakito pour ton grand coeur, Nicou pour ton flow et tes serrages. Merci Sara de ne pas me détester. Xie xie Yanan for the hot pot.

% Amis et famille
Je remercie aussi du fond du coeur ma famille, ma belle-famille et mes amis. Merci pour votre amour et votre indéfectible soutien qui m'ont permis d'avancer jusqu'ici et de devenir qui je suis.

% Chloé et les enfants
Last but not least, je remercie ma compagne, Chloé. Merci pour ton oreille attentive et bienveillante surtout dans mes moments d'errance quand j'envisageais une reconversion en tant que jardinier ou éleveur de chèvres en Suisse. Je te remercie aussi d'être celle qui m'accompagne au quotidien depuis toutes ces années, et, surtout, d'être la formidable maman de nos deux enfants, Victor et Oscar, et de me permettre ainsi d'exercer le plus beau métier du monde.

% Autres
Ci-dessus n'était que l'humble expression de ma profonde gratitude envers celleux qui m'ont accompagné tout au long de ce chemin tantôt long fleuve tranquille tantôt parcours semé d'embûches. J'ai certainement omis de mentionner nombreux d'entre vous. Sachez cependant que sans votre présence et votre soutien, ce beau projet professionnel mais, avant tout, cette fabuleuse aventure personnelle n'aurait pas été possible.




