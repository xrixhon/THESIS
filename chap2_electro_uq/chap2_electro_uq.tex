%!TEX root = ../thesis_main.tex
%!TEX encoding = UTF-8 Unicode

In this chapter aims at showing the first step of uncertainty quantification and highlight a first insight in terms of impacting parameters. It'd also highlight that depending on the gwp\_limit 
\section{From a cost-optimised to a carbon-neutral Belgian energy system: progressive defossilisation}
\label{sec:chap_2_ses_sec2}
In this section, I'd include the section 2 of the electrofuels+UQ paper (\url{https://www.mdpi.com/1996-1073/14/13/4027})(without 2.2 where I presented EnergyScope that I'd include in Chapter 1).

\subsection{Electrofuels}
\label{subsec:chap2_electrofuels}
This section would be as the section 2.1 of the electrofuels+UQ paper (\url{https://www.mdpi.com/1996-1073/14/13/4027}), where I present more specficially how electrofuels are implemented in the model

\subsection{Reference Case Study: The Belgian Energy System in 2050}
\label{subsec:chap2_case_study}
Basically section 2.3 of the electrofuels+UQ paper (\url{https://www.mdpi.com/1996-1073/14/13/4027}), where I present more specficially how electrofuels are implemented in the model where I present what a cost-optimised (without gwp-constraint) Belgian energy system would look like

\subsection{A step by step defossilisation of the snapshot system}
\label{subsec:chap2_defossilisation}
As in section 2.4, quickly show how we implement the defossilisation without having an entire pathway.

\section{Results}
\label{sec:chap2_results}
Similar to Section 4 of the electrofuels+UQ paper (\url{https://www.mdpi.com/1996-1073/14/13/4027}). There'll be work to generate up-to-date results as the model evolved (no more FC\_cars but BEV, for instance)
\subsection{Statistical analysis of the cost}
\label{subsec:chap_2_stat_analysis}

\subsection{Critical parameters}
\label{subsec:chap_2_crit_param}

\section{Discussion and perspective with the literature}
Similar to section 5.1 of the electrofuels+UQ paper (\url{https://www.mdpi.com/1996-1073/14/13/4027})
