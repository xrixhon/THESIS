%!TEX root = ../thesis_main.tex
%!TEX encoding = UTF-8 Unicode
In this chapter, we introduce the two main tools, not exhaustively of course (because already presented, in Gauthier and Diederik's theses) on which I based my work: EnergyScope and RHEIA/PCE. I'd like too to talk about the case study (Belgian energy system) that will be addressed all along the manuscript but I'm afraid it'd be too many stuff in one chapter. On the other side, I would not dedicate an entire chapter to any of these three parts. What's your opinion?
\section{EnergyScope: To optimise the energy transition pathway}
\label{sec:energyscope}

\subsection{EnergyScope TD}
\label{subsec:estd}
Presentation of the capacity to model a whole-energy system,with a hourly resolution, and of the main equations of the snapshot model (similarly to what I've done in the electrofuels+UQ paper (\url{https://www.mdpi.com/1996-1073/14/13/4027}).

\subsection{EnergyScope Pathway}
\label{subsec:espathway}
From snapshot to pathway optimisation, presentation of the main equations to link the representative years, focus on the salvage value (not presented in Gauthier's thesis but well in our Pathway paper)

\section{RHEIA: To quantify the uncertainties}
\label{sec:rheia}


\subsection{Uncertainty characterisation}
\label{subsec:uncert_charac}
Presentation here of the uncertainty characterisation from S. Moret and adding some parameters specific to the pathway model (\eg $\Delta_{\text{change}}$ linked to change speed) or specific to the main case study (\eg possibility to have nuclear SMR from 2040)


\subsection{Polynomial Chaos Expansion}
\label{subsec:pce}
Similarly to what we did in the electrofuels+UQ paper (\url{https://www.mdpi.com/1996-1073/14/13/4027}), presentation of the PCE and Sobol' sequence (which is a more optimised way to explore the ranges of uncertainties, compared, for instance, to random exploration).


\subsection{Preliminary screening and selection}
\label{subsec:screening}
Similarly to what we did in the electrofuels+UQ paper (\url{https://www.mdpi.com/1996-1073/14/13/4027}), emphasise that we only consider a limited amount of uncertain parameters to keep a reasonable computation time while capturing the impact of (almost) all the uncertainties.

\section{Case study: The Belgian energy system}
\label{sec:case_study}
Similarly to what we did in the electrofuels+UQ paper (\url{https://www.mdpi.com/1996-1073/14/13/4027}),I'd present generally here the case study, much shorter than what Gauthier did in his thesis. It'd be presenting the different sets of resources, technologies and demands.

Here, I could integrate the information on the NED and how important it is to consider it as it represents 10\% worldwide (and even 20\% in Belgium) of the energy consumption, especially because it is usually a sector that is overlooked in whole-energy system optimisation.









