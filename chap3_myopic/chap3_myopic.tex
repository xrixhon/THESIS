%!TEX root = ../thesis_main.tex
%!TEX encoding = UTF-8 Unicode
In this chapter, I highlight the advantages of the myopic pathway, the methodological adaptations needed to have a myopic optimisation from EnergyScope Pathway perfect foresight and detail the difference with the perfect foresight for the case study. 

\section{Why and how myopic}
\label{sec:chap3_why_how}
Besides the fact that it is a necessary framework for the application of RL,  explain here why it is interesting, as is, to consider myopic optimisation (time saving, more appropriate to mimic policymakers' shortsightedness, etc.)

Detail the little tweeks to go from a perfect foresight to a myopic optimisation, in terms of implementation.

\section{Case study and results: Belgian energy system under \ce{CO2} trajectory}
\label{sec:chap3_case_study_results}

\subsection{Case study}
{\color{red}Question to HJ and FC : J'hésite sur le cas de référence pour la comparaison en termes de trajectoire \ce{CO2}. Voici ce à quoi je pense:
\begin{enumerate}
\item Comme ce qu'on a présenté dans le papier Pathway, une décroissance linéaire entre les émissions de 2020 et la neutralité carbone en 2050
\item Sur base du budget \ce{CO2} identique à celui de la décroissance linéaire (~1.9Gt\ce{CO2}), imposer au myopique la trajectoire \ce{CO2} issue de l'optimisation perfect foresight. Dans cette option, il y a deux sous-choix:
\begin{itemize}
\item Imposer la neutralité carbone en 2050, comme dans le cas 1.
\item Ne pas imposer la neutralité carbone en 2050
\end{itemize}
\item Sur base du budget \ce{CO2} que je donne implicitement comme objectif à mon agent (~1.2Gt\ce{CO2}), imposer au myopique la trajectoire \ce{CO2} issue de l'optimisation perfect foresight. Dans cette option, il y a deux sous-choix:
\begin{itemize}
\item Imposer la neutralité carbone en 2050, comme dans le cas 1.
\item Ne pas imposer la neutralité carbone en 2050
\end{itemize}
\end{enumerate}
Perso, je choisirais de me rapprocher le plus possible de ce que je fais ensuite avec l'agent. Du coup, ce serait le cas 3  sans imposer la neutralité carbone 2050. Dans ce cas, pour l'avoir observé, le modèle n'atteint pas la neutralité carbone. 

Quel est votre avis sur la question?
}

Define the case study and the \ce{CO2} trajectory. 

\subsection{Results and comparison with Perfect foresight}

\subsection{Discussion and perspective with the literature} 